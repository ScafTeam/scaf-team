\documentclass{report}
\usepackage[margin=2cm, bottom=1.5cm]{geometry}
\usepackage{type1cm}
\usepackage{amssymb}
\usepackage[fleqn]{amsmath}
\usepackage{tikz}
\usepackage{multicol}
\usepackage{makecell}
\usepackage{tabularx}
\usepackage[shortlabels]{enumitem}
\setlength{\columnsep}{1pt}
\setlist[enumerate]{nosep}

\makeatletter
\newenvironment{myalign*}{\ifvmode\else\hfil\null\linebreak\fi
  \hspace*{-\leftmargin}\minipage\textwidth
  \setlength{\abovedisplayskip}{0pt}%
  \setlength{\abovedisplayshortskip}{\abovedisplayskip}%
  \start@align\@ne\st@rredtrue\m@ne}%
{\endalign\endminipage\linebreak}

\usepackage{xeCJK}
\setCJKmainfont{標楷體}

\begin{document}
\title{%
  \fontsize{40}{60}\selectfont
  SCAF  \\ % 
  \vspace*{2cm}%
  \fontsize{24}{30}\selectfont
  測試文件
}

\author{
  \fontsize{18}{28}\selectfont
  \begin{tabularx}{\textwidth}{
    |p{\dimexpr.25\linewidth-2\tabcolsep-1.33333\arrayrulewidth}%
    |p{\dimexpr.75\linewidth-2\tabcolsep-1.33333\arrayrulewidth}|%
  }
    \hline
    \centering 專案名稱 & SCAF 開發輔助工具 \\
    \hline
    \centering 撰寫日期 & 2022 / 12 / 22 \\
    \hline
    \centering 發展者 & 簡蔚驊 \! 鄧暐宣 \! 余威霆 \! 林佳何 \! 唐劭賢 \\
    \hline
  \end{tabularx}
}
\date{}
\usetikzlibrary{automata, positioning, arrows}
\maketitle
\tikzset{every state, accepting/.style={double distance=2pt}}

\fontsize{12}{18}\selectfont
\section*{1. 接測試目的與接受準則(Objectives and Acceptance Criteria)}
\subsection*{1.1 系統範圍(System Scope)}
  \begin{itemize}
    \item 完成進度
  \end{itemize}

\subsection*{1.2 測試接受準則(Test Acceptance Criteria)}
本測試計劃需要滿足下面的測試接受準則:\\
測試程序需要依照本測試計劃所訂定的程序進行,所有測試結果需要能符合預期測試結果方能接受。
  \begin{itemize}
    \item 當測試案例未通過時,相關模組開發之負責人需要進行程式修改(修復bug或改動功能),以能讓此案例重新通過測試。
    \item 重新進行測試時,測試人員需確認其他可能受影響的案例仍可正確執行。
  \end{itemize}


\section*{2. 測試環境(Testing Environment)}
\subsection*{2.1硬體需求(Hardware Specification and Configuration)}
\begin{tabularx}{\textwidth}{
  |p{\dimexpr.1\linewidth-2\tabcolsep-1.33333\arrayrulewidth}%
  |p{\dimexpr.3\linewidth-2\tabcolsep-1.33333\arrayrulewidth}%
  |p{\dimexpr.1\linewidth-2\tabcolsep-1.33333\arrayrulewidth}%
  |p{\dimexpr.3\linewidth-2\tabcolsep-1.33333\arrayrulewidth}%
  |p{\dimexpr.2\linewidth-2\tabcolsep-1.33333\arrayrulewidth}|%
}
  \hline
  項次 &  名稱 & 數量 & 規格  & 備註  \\ \hline
  1 & TUF Gaming FX507ZM & 1 & 16G RAM, 512G SSD & window系統  \\ \hline
  2 & ROG Zephyrus GA503QS & 1 & 16G RAM, 512G SSD & window系統  \\ \hline
  3 & ThinkPad E15 Gen 4 & 1 & 8G RAM, 512G SSD & window系統  \\ \hline
  4 & ThinkPad E15 Gen 4 & 1 & 16G RAM, 512G SSD & window系統  \\ \hline
  
\end{tabularx}

\subsection*{2.2軟體需求(Hardware Specification and Configuration)}
\begin{tabularx}{\textwidth}{ 
  |p{\dimexpr.1\linewidth-2\tabcolsep-1.33333\arrayrulewidth}%
  |p{\dimexpr.3\linewidth-2\tabcolsep-1.33333\arrayrulewidth}%
  |p{\dimexpr.1\linewidth-2\tabcolsep-1.33333\arrayrulewidth}%
  |p{\dimexpr.3\linewidth-2\tabcolsep-1.33333\arrayrulewidth}%
  |p{\dimexpr.2\linewidth-2\tabcolsep-1.33333\arrayrulewidth}|%
}
  \hline
  項次 &  名稱 & 數量 & 規格  & 備註  \\
  \hline
  1 & chrome & 1 & 版本 108.0.5359.125 & clinet端瀏覽器  \\
  \hline
  2 & Firebase & 1 & 版本 11.4.0 & server端資料庫  \\
  \hline
\end{tabularx}

\subsection*{2.3測試資料來源(Test Data Sources)}
自行開發測試資料


\subsection*{2.4測試工具與設備(Tools and Equipment)}
Postman、Jmeter

\section*{3. 測試案例(Test Cases)}
\begin{tabularx}{\textwidth}{
  |p{\dimexpr.25\linewidth-2\tabcolsep-1.33333\arrayrulewidth}%
  |p{\dimexpr.75\linewidth-2\tabcolsep-1.33333\arrayrulewidth}|%
  }
  \hline
  \centering Identification &  SCAF-TC-01 \\
  \hline
  \centering Name & 登入帳號測試 \\
  \hline
  \centering Reference & SCAF-FR-01 \\
  \hline
  \centering Severity & 高 \\
  \hline
  \centering Instructions & 
  \makecell[l]{
    1. 輸入網址進入SCAF系統 \\
    2. 輸入已註冊過的使用者信箱及密碼 \\
    3. 點擊 Sign in
  }\\
  \hline
  \centering Expected result & 成功登入系統,並能在setting頁面看到正確使用者資訊 \\
  \hline
  \centering Cleanup & 無 \\
  \hline
\end{tabularx}
\newline\newline
\begin{tabularx}{\textwidth}{
  |p{\dimexpr.25\linewidth-2\tabcolsep-1.33333\arrayrulewidth}%
  |p{\dimexpr.75\linewidth-2\tabcolsep-1.33333\arrayrulewidth}|%
  }
  \hline
  \centering Identification &  SCAF-TC-02 \\
  \hline
  \centering Name & 登入帳號測試 \\
  \hline
  \centering Reference & SCAF-FR-01 \\
  \hline
  \centering Severity & 高 \\
  \hline
  \centering Instructions & 
  \makecell[l]{
    1. 輸入網址進入SCAF系統 \\
    2. 輸入存在可未註冊的使用者信箱及密碼 EX: Gmail + 密碼 \\
    3. 點擊 Sign in
  }\\
  \hline
  \centering Expected result & 系統顯示錯誤的使用者名稱或密碼 \\
  \hline
  \centering Cleanup & 無 \\
  \hline
\end{tabularx}
\\
\newline
\\
\begin{tabularx}{\textwidth}{
  |p{\dimexpr.25\linewidth-2\tabcolsep-1.33333\arrayrulewidth}%
  |p{\dimexpr.75\linewidth-2\tabcolsep-1.33333\arrayrulewidth}|%
  }
  \hline
  \centering Identification &  SCAF-TC-03 \\
  \hline
  \centering Name & 註冊帳號測試 \\
  \hline
  \centering Reference & SCAF-FR-02 \\
  \hline
  \centering Severity & 高 \\
  \hline
  \centering Instructions & 
  \makecell[l]{
    1. 在登入頁面點選Sign Up進入註冊頁面 \\
    2. 輸入存在可未註冊的使用者信箱及密碼 EX: Gmail + 密碼 \\
    3. 點擊 Sign Up
    4. 輸入剛註冊完的信箱及密碼
    5. 點擊 Sign in
  }\\
  \hline
  \centering Expected result & 系統顯示註冊成功,並成功登入系統 \\
  \hline
  \centering Cleanup & 無 \\
  \hline
\end{tabularx}
\\
\newline
\\
\begin{tabularx}{\textwidth}{
  |p{\dimexpr.25\linewidth-2\tabcolsep-1.33333\arrayrulewidth}%
  |p{\dimexpr.75\linewidth-2\tabcolsep-1.33333\arrayrulewidth}|%
  }
  \hline
  \centering Identification &  SCAF-TC-04 \\
  \hline
  \centering Name & 註冊帳號測試 \\
  \hline
  \centering Reference & SCAF-FR-02 \\
  \hline
  \centering Severity & 中 \\
  \hline
  \centering Instructions & 
  \makecell[l]{
    1. 在登入頁面點選Sign Up進入註冊頁面 \\
    2. 輸入不符合格式的信箱或已被註冊過的信箱  \\
    3. 點擊 Sign Up
  }\\
  \hline
  \centering Expected result & 系統顯示註冊失敗,要求使用者重新輸入 \\
  \hline
  \centering Cleanup & 無 \\
  \hline
\end{tabularx}
\\
\newline
\\
\begin{tabularx}{\textwidth}{
  |p{\dimexpr.25\linewidth-2\tabcolsep-1.33333\arrayrulewidth}%
  |p{\dimexpr.75\linewidth-2\tabcolsep-1.33333\arrayrulewidth}|%
  }
  \hline
  \centering Identification &  SCAF-TC-05 \\
  \hline
  \centering Name & 忘記密碼測試 \\
  \hline
  \centering Reference & SCAF-FR-03 \\
  \hline
  \centering Severity & 高 \\
  \hline
  \centering Instructions & 
  \makecell[l]{
    1. 在登入頁面點選Forgot Password?進入重設密碼頁面 \\
    2. 輸入已註冊過的信箱  \\
    3. 點擊Submit \\
    4. 在信箱中確認Firebase發送的重設密碼信件 \\
    5. 輸入新的密碼 \\
    6. 回到登入頁面輸入更改後的密碼  \\
    7. 點擊 Sign in
  }\\
  \hline
  \centering Expected result & 在信箱中成功接收Firebase的信件,並在修改過後成功登入系統 \\
  \hline
  \centering Cleanup & 無 \\
  \hline
\end{tabularx}
\\
\newline
\\
\begin{tabularx}{\textwidth}{
  |p{\dimexpr.25\linewidth-2\tabcolsep-1.33333\arrayrulewidth}%
  |p{\dimexpr.75\linewidth-2\tabcolsep-1.33333\arrayrulewidth}|%
  }
  \hline
  \centering Identification &  SCAF-TC-06 \\
  \hline
  \centering Name & 忘記密碼測試 \\
  \hline
  \centering Reference & SCAF-FR-03 \\
  \hline
  \centering Severity & 中 \\
  \hline
  \centering Instructions & 
  \makecell[l]{
    1. 在登入頁面點選Forgot Password?進入重設密碼頁面 \\
    2. 輸入未註冊過的信箱  \\
    3. 點擊Submit
    4. 在信箱中確認Firebase發送的重設密碼信件
  }\\
  \hline
  \centering Expected result & 確認信箱中未出現信件 \\
  \hline
  \centering Cleanup & 無 \\
  \hline
\end{tabularx}
\\
\newline
\\
\begin{tabularx}{\textwidth}{
  |p{\dimexpr.25\linewidth-2\tabcolsep-1.33333\arrayrulewidth}%
  |p{\dimexpr.75\linewidth-2\tabcolsep-1.33333\arrayrulewidth}|%
  }
  \hline
  \centering Identification &  SCAF-TC-07 \\
  \hline
  \centering Name & 共用專案測試 \\
  \hline
  \centering Reference & SCAF-FR-04 \\
  \hline
  \centering Severity & 中 \\
  \hline
  \centering Instructions & 
  \makecell[l]{
    1. 專案擁有者點擊My project到專案列表頁面 \\
    2. 點選隨意一個專案進入到專案中 \\
    3. 專案擁有者點擊Project名稱下方的Setting \\
    4. 輸入已註冊過的使用者信箱 \\
    5. 接收者點擊通知中的專案邀請訊息 \\
    6. 接收者點擊確認加入專案 \\
    7. 接收者點擊navrbar的My project回到專案列表 \\
    8. 進入專案查看其他專案資訊
  }\\
  \hline
  \centering Expected result & 接收者的專案列表中,成功出現接收邀請的專案,且專案內容皆與擁有者相同 \\
  \hline
  \centering Cleanup & 無 \\
  \hline
\end{tabularx}
\\
\newline
\\
\begin{tabularx}{\textwidth}{
  |p{\dimexpr.25\linewidth-2\tabcolsep-1.33333\arrayrulewidth}%
  |p{\dimexpr.75\linewidth-2\tabcolsep-1.33333\arrayrulewidth}|%
  }
  \hline
  \centering Identification &  SCAF-TC-08 \\
  \hline
  \centering Name & 更改專案內容測試 \\
  \hline
  \centering Reference & SCAF-FR-04 \\
  \hline
  \centering Severity & 高 \\
  \hline
  \centering Instructions & 
  \makecell[l]{
    1. 點擊My project到專案列表頁面 \\
    2. 點選隨意一個專案進入到專案中 \\
    3. 點擊專案名稱下方的documnet \\
    4. 編輯文件內容 \\
    5. 點擊Save
  }\\
  \hline
  \centering Expected result & 其餘專案成員皆能看見檔案的變更 \\
  \hline
  \centering Cleanup & 無 \\
  \hline
\end{tabularx}
\\
\newline
\\
\begin{tabularx}{\textwidth}{
  |p{\dimexpr.25\linewidth-2\tabcolsep-1.33333\arrayrulewidth}%
  |p{\dimexpr.75\linewidth-2\tabcolsep-1.33333\arrayrulewidth}|%
  }
  \hline
  \centering Identification &  SCAF-TC-09 \\
  \hline
  \centering Name & 刪除專案測試 \\
  \hline
  \centering Reference & SCAF-FR-04 \\
  \hline
  \centering Severity & 高 \\
  \hline
  \centering Instructions & 
  \makecell[l]{
    1. 專案擁有者點擊My project到專案列表頁面 \\
    2. 專案擁有者點擊專案的X進行刪除專案 \\
    3. 專案內的成員查看專案列表頁面 
  }\\
  \hline
  \centering Expected result & 專案內的所有成員此專案皆被刪除(未再顯示在專案列表上) \\
  \hline
  \centering Cleanup & 無 \\
  \hline
\end{tabularx}
\\
\newline
\\
\begin{tabularx}{\textwidth}{
  |p{\dimexpr.25\linewidth-2\tabcolsep-1.33333\arrayrulewidth}%
  |p{\dimexpr.75\linewidth-2\tabcolsep-1.33333\arrayrulewidth}|%
  }
  \hline
  \centering Identification &  SCAF-TC-10 \\
  \hline
  \centering Name & 刪除專案測試 \\
  \hline
  \centering Reference & SCAF-FR-04 \\
  \hline
  \centering Severity & 中 \\
  \hline
  \centering Instructions & 
  \makecell[l]{
    1. 專案成員點擊My project到專案列表頁面 \\
    2. 專案成員點擊專案的X進行刪除專案 \\
    3. 專案內的成員查看專案列表頁面  
  }\\
  \hline
  \centering Expected result & 系統顯示使用者並非專案擁有者不可刪除專案,且專案成員的專案列表上此專案並未被刪除 \\
  \hline
  \centering Cleanup & 無 \\
  \hline
\end{tabularx}
\\
\newline
\\
\begin{tabularx}{\textwidth}{
  |p{\dimexpr.25\linewidth-2\tabcolsep-1.33333\arrayrulewidth}%
  |p{\dimexpr.75\linewidth-2\tabcolsep-1.33333\arrayrulewidth}|%
  }
  \hline
  \centering Identification &  SCAF-TC-11 \\
  \hline
  \centering Name & 開發流程測試 \\
  \hline
  \centering Reference & SCAF-FR-05 \\
  \hline
  \centering Severity & 低 \\
  \hline
  \centering Instructions & 
  \makecell[l]{
    1. 專案擁有者點擊Project名稱下方的Setting \\
    2. 專案成員選擇瀑布式開發或敏捷式開發 \\
  }\\
  \hline
  \centering Expected result & 系統跳出兩個流程的的執行方法,並舉例設定流程日期的方式 \\
  \hline
  \centering Cleanup & 無 \\
  \hline
\end{tabularx}
\\
\newline
\\
\begin{tabularx}{\textwidth}{
  |p{\dimexpr.25\linewidth-2\tabcolsep-1.33333\arrayrulewidth}%
  |p{\dimexpr.75\linewidth-2\tabcolsep-1.33333\arrayrulewidth}|%
  }
  \hline
  \centering Identification &  SCAF-TC-12 \\
  \hline
  \centering Name & 專案初始化測試 \\
  \hline
  \centering Reference & SCAF-FR-06 \\
  \hline
  \centering Severity & 中 \\
  \hline
  \centering Instructions & 
  \makecell[l]{
    1. 點擊My project到專案列表頁面  \\
    2. 點擊+號創建一個新的專案  \\
    3. 輸入名稱、選擇加入README檔  \\
    4. 按下Create
  }\\
  \hline
  \centering Expected result & 專案內document有README檔、專案列表有新增該檔案 \\
  \hline
  \centering Cleanup & 無 \\
  \hline
\end{tabularx}
\\
\newline
\\
\begin{tabularx}{\textwidth}{
  |p{\dimexpr.25\linewidth-2\tabcolsep-1.33333\arrayrulewidth}%
  |p{\dimexpr.75\linewidth-2\tabcolsep-1.33333\arrayrulewidth}|%
  }
  \hline
  \centering Identification &  SCAF-TC-13 \\
  \hline
  \centering Name & 專案初始化測試 \\
  \hline
  \centering Reference & SCAF-FR-06 \\
  \hline
  \centering Severity & 中 \\
  \hline
  \centering Instructions & 
  \makecell[l]{
    1. 點擊My project到專案列表頁面  \\
    2. 點擊+號創建一個新的專案  \\
    3. 輸入名稱、不選擇加入README檔  \\
    4. 按下Create
  }\\
  \hline
  \centering Expected result & 專案內document沒有README檔、專案列表有新增該檔案 \\
  \hline
  \centering Cleanup & 無 \\
  \hline
\end{tabularx}
\\
\newline
\\
\begin{tabularx}{\textwidth}{
  |p{\dimexpr.25\linewidth-2\tabcolsep-1.33333\arrayrulewidth}%
  |p{\dimexpr.75\linewidth-2\tabcolsep-1.33333\arrayrulewidth}|%
  }
  \hline
  \centering Identification &  SCAF-TC-14 \\
  \hline
  \centering Name & 新增repository \\
  \hline
  \centering Reference & SCAF-FR-07 \\
  \hline
  \centering Severity & 高 \\
  \hline
  \centering Instructions & 
  \makecell[l]{
    1. 點擊專案列表隨意一個專案 \\
    2. 點擊專案名稱下方的repositories \\
    3. 點擊+號創建一個新的repository \\
    3. 輸入repository 名稱、對應的github網址  \\
    4. 按下Create
  }\\
  \hline
  \centering Expected result & repository列表應新增此repository,點擊連結能夠跳轉到對應的網址 \\
  \hline
  \centering Cleanup & 無 \\
  \hline
\end{tabularx}
\\
\newline
\\
\begin{tabularx}{\textwidth}{
  |p{\dimexpr.25\linewidth-2\tabcolsep-1.33333\arrayrulewidth}%
  |p{\dimexpr.75\linewidth-2\tabcolsep-1.33333\arrayrulewidth}|%
  }
  \hline
  \centering Identification &  SCAF-TC-15 \\
  \hline
  \centering Name & 新增repository \\
  \hline
  \centering Reference & SCAF-FR-07 \\
  \hline
  \centering Severity & 低 \\
  \hline
  \centering Instructions & 
  \makecell[l]{
    1. 點擊專案列表隨意一個專案 \\
    2. 點擊專案名稱下方的repositories \\
    3. 點擊+號創建一個新的repository \\
    3. 輸入repository 名稱、不存在的github網址  \\
    4. 按下Create
  }\\
  \hline
  \centering Expected result & 系統顯示此連結網址無效,請重新輸入 \\
  \hline
  \centering Cleanup & 無 \\
  \hline
\end{tabularx}
\\
\newline
\\
\begin{tabularx}{\textwidth}{
  |p{\dimexpr.25\linewidth-2\tabcolsep-1.33333\arrayrulewidth}%
  |p{\dimexpr.75\linewidth-2\tabcolsep-1.33333\arrayrulewidth}|%
  }
  \hline
  \centering Identification &  SCAF-TC-16 \\
  \hline
  \centering Name & 刪除repository \\
  \hline
  \centering Reference & SCAF-FR-07 \\
  \hline
  \centering Severity & 高 \\
  \hline
  \centering Instructions & 
  \makecell[l]{
    1. 點擊專案名稱下方的repositories \\
    2. 點擊repository旁邊的X刪除repository \\
  }\\
  \hline
  \centering Expected result & 此repository在repositories列表中消失 \\
  \hline
  \centering Cleanup & 無 \\
  \hline
\end{tabularx}
\\
\newline
\\
\begin{tabularx}{\textwidth}{
  |p{\dimexpr.25\linewidth-2\tabcolsep-1.33333\arrayrulewidth}%
  |p{\dimexpr.75\linewidth-2\tabcolsep-1.33333\arrayrulewidth}|%
  }
  \hline
  \centering Identification &  SCAF-TC-17 \\
  \hline
  \centering Name & 預覽文件測試 \\
  \hline
  \centering Reference & SCAF-FR-08 \\
  \hline
  \centering Severity & 中 \\
  \hline
 \centering Instructions & 
  \makecell[l]{
    1. 點擊專案名稱下方的documnet \\
    2. 在畫面中瀏覽已存在的檔案
  }\\
  \hline
  \centering Expected result & 檔案呈現唯獨狀態,顯示在頁面上 \\
  \hline
  \centering Cleanup & 無 \\
  \hline
\end{tabularx}
\\
\newline
\\

\begin{tabularx}{\textwidth}{
  |p{\dimexpr.25\linewidth-2\tabcolsep-1.33333\arrayrulewidth}%
  |p{\dimexpr.75\linewidth-2\tabcolsep-1.33333\arrayrulewidth}|%
  }
  \hline
  \centering Identification &  SCAF-TC-18 \\
  \hline
  \centering Name & 編輯文件測試 \\
  \hline
  \centering Reference & SCAF-FR-08 \\
  \hline
  \centering Severity & 高 \\
  \hline
  \centering Instructions & 
  \makecell[l]{
    1. 點擊documnet頁面中需求文件 \\
    2. 更改該文件內容  \\
    3. 點擊Save 
  }\\
  \hline
  \centering Expected result & 其他使用者頁面應顯示更動內容 \\
  \hline
  \centering Cleanup & 無 \\
  \hline
\end{tabularx}
\\
\newline
\\
\begin{tabularx}{\textwidth}{
  |p{\dimexpr.25\linewidth-2\tabcolsep-1.33333\arrayrulewidth}%
  |p{\dimexpr.75\linewidth-2\tabcolsep-1.33333\arrayrulewidth}|%
  }
  \hline
  \centering Identification &  SCAF-TC-19 \\
  \hline
  \centering Name & 編輯文件測試 \\
  \hline
  \centering Reference & SCAF-FR-08 \\
  \hline
  \centering Severity & 低 \\
  \hline
  \centering Instructions & 
  \makecell[l]{
    1. 點擊documnet頁面中需求文件 \\
    2. 更改該文件內容  \\
    3. 點擊其他頁面 
  }\\
  \hline
  \centering Expected result & 預覽頁面應顯示未修改的畫面 \\
  \hline
  \centering Cleanup & 無 \\
  \hline
\end{tabularx}
\\
\newline
\\
\begin{tabularx}{\textwidth}{
  |p{\dimexpr.25\linewidth-2\tabcolsep-1.33333\arrayrulewidth}%
  |p{\dimexpr.75\linewidth-2\tabcolsep-1.33333\arrayrulewidth}|%
  }
  \hline
  \centering Identification &  SCAF-TC-20 \\
  \hline
  \centering Name & 新增一個任務至看板中 \\
  \hline
  \centering Reference & SCAF-FR-09 \\
  \hline
  \centering Severity & 高 \\
  \hline
  \centering Instructions & 
  \makecell[l]{
    1. 點擊專案名稱下方的Kanban  \\
    2. 按下 TO DO 旁邊的+  \\
    3. 填入任務名稱、任務敘述  \\
    4. 按下Create
  }\\
  \hline
  \centering Expected result & TO DO下方應增加一個任務 \\
  \hline
  \centering Cleanup & 無 \\
  \hline
\end{tabularx}
\\
\newline
\\
\begin{tabularx}{\textwidth}{
  |p{\dimexpr.25\linewidth-2\tabcolsep-1.33333\arrayrulewidth}%
  |p{\dimexpr.75\linewidth-2\tabcolsep-1.33333\arrayrulewidth}|%
  }
  \hline
  \centering Identification &  SCAF-TC-21 \\
  \hline
  \centering Name & 新增一個任務至看板中 \\
  \hline
  \centering Reference & SCAF-FR-09 \\
  \hline
  \centering Severity & 低 \\
  \hline
  \centering Instructions & 
  \makecell[l]{
    1. 點擊專案名稱下方的Kanban  \\
    2. 按下 TO DO 旁邊的+  \\
    3. 填入任務名稱、不填入任務敘述  \\
    4. 按下Create
  }\\
  \hline
  \centering Expected result & TO DO下方應增加一個任務(無描述) \\
  \hline
  \centering Cleanup & 無 \\
  \hline
\end{tabularx}
\\
\newline
\\
\begin{tabularx}{\textwidth}{
  |p{\dimexpr.25\linewidth-2\tabcolsep-1.33333\arrayrulewidth}%
  |p{\dimexpr.75\linewidth-2\tabcolsep-1.33333\arrayrulewidth}|%
  }
  \hline
  \centering Identification &  SCAF-TC-22 \\
  \hline
  \centering Name & 移動看板中的任務 \\
  \hline
  \centering Reference & SCAF-FR-09 \\
  \hline
  \centering Severity & 高 \\
  \hline
  \centering Instructions & 
  \makecell[l]{
    1. 拖曳Doing下方任一個任務  \\
    2. 在Done的位置放開滑鼠 
  }\\
  \hline
  \centering Expected result & 此任務消失在Doing欄位,並出現在Done \\
  \hline
  \centering Cleanup & 無 \\
  \hline
\end{tabularx}
\\
\newline
\\
\begin{tabularx}{\textwidth}{
  |p{\dimexpr.25\linewidth-2\tabcolsep-1.33333\arrayrulewidth}%
  |p{\dimexpr.75\linewidth-2\tabcolsep-1.33333\arrayrulewidth}|%
  }
  \hline
  \centering Identification &  SCAF-TC-23 \\
  \hline
  \centering Name & 移動看板中的任務 \\
  \hline
  \centering Reference & SCAF-FR-09 \\
  \hline
  \centering Severity & 低 \\
  \hline
  \centering Instructions & 
  \makecell[l]{
    1. 拖曳Doing下方任一個任務  \\
    2. 在Doing的位置放開滑鼠  \\
  }\\
  \hline
  \centering Expected result & 此頁面應無任何改變 \\
  \hline
  \centering Cleanup & 無 \\
  \hline
\end{tabularx}
\\
\newline
\\
\begin{tabularx}{\textwidth}{
  |p{\dimexpr.25\linewidth-2\tabcolsep-1.33333\arrayrulewidth}%
  |p{\dimexpr.75\linewidth-2\tabcolsep-1.33333\arrayrulewidth}|%
  }
  \hline
  \centering Identification &  SCAF-TC-24 \\
  \hline
  \centering Name & 接收通知 \\
  \hline
  \centering Reference & SCAF-FR-10 \\
  \hline
  \centering Severity & 低 \\
  \hline
  \centering Instructions & 
  \makecell[l]{
    1. 點擊documnet頁面中需求文件的edit \\
    2. 更改該文件內容  \\
    3. 存檔  \\
    4. 點擊Navrbar中的鈴鐺
  }\\
  \hline
  \centering Expected result & 需求文件更動的通知應顯示 \\
  \hline
  \centering Cleanup & 無 \\
  \hline
\end{tabularx}
\\
\newline
\\
\begin{tabularx}{\textwidth}{
  |p{\dimexpr.25\linewidth-2\tabcolsep-1.33333\arrayrulewidth}%
  |p{\dimexpr.75\linewidth-2\tabcolsep-1.33333\arrayrulewidth}|%
  }
  \hline
  \centering Identification &  SCAF-TC-25 \\
  \hline
  \centering Name & 專案邀請 \\
  \hline
  \centering Reference & SCAF-FR-10 \\
  \hline
  \centering Severity & 中 \\
  \hline
  \centering Instructions & 
  \makecell[l]{
    1. 專案擁有者點擊My project到專案列表頁面 \\
    2. 點選隨意一個專案進入到專案中 \\
    3. 專案擁有者點擊Project名稱下方的Setting \\
    4. 輸入已註冊過的使用者信箱 \\
    5. 接收者點擊Navrbar中的鈴鐺 \\
    6. 點擊專案邀請的訊息 
  }\\
  \hline
  \centering Expected result & 顯示專案名稱、邀請人、是否加入的選項 \\
  \hline
  \centering Cleanup & 無 \\
  \hline
\end{tabularx}
\\
\newline
\\
\begin{tabularx}{\textwidth}{
  |p{\dimexpr.25\linewidth-2\tabcolsep-1.33333\arrayrulewidth}%
  |p{\dimexpr.75\linewidth-2\tabcolsep-1.33333\arrayrulewidth}|%
  }
  \hline
  \centering Identification &  SCAF-TC-26 \\
  \hline
  \centering Name & 變更名稱、自介 \\
  \hline
  \centering Reference & SCAF-FR-11 \\
  \hline
  \centering Severity & 低 \\
  \hline
  \centering Instructions & 
  \makecell[l]{
    1. 點擊Navrbar中的Setting \\
    2. 填入新的名稱、自介 \\
    3. 點擊Save \\
    3. 重新點擊Setting頁面
  }\\
  \hline
  \centering Expected result & 顯示修改後的資料 \\
  \hline
  \centering Cleanup & 無 \\
  \hline
\end{tabularx}
\\
\newline
\\
\begin{tabularx}{\textwidth}{
  |p{\dimexpr.25\linewidth-2\tabcolsep-1.33333\arrayrulewidth}%
  |p{\dimexpr.75\linewidth-2\tabcolsep-1.33333\arrayrulewidth}|%
  }
  \hline
  \centering Identification &  SCAF-TC-27 \\
  \hline
  \centering Name & 變更密碼 \\
  \hline
  \centering Reference & SCAF-FR-11 \\
  \hline
  \centering Severity & 中 \\
  \hline
  \centering Instructions & 
  \makecell[l]{
    1. 點擊Navrbar中的Setting \\
    2. 填入舊密碼
    3. 新密碼 \\
    4. 點擊Save \\
    5. 登出系統 \\
    6. 重新登入系統
  }\\
  \hline
  \centering Expected result & 成功登入系統 \\
  \hline
  \centering Cleanup & 無 \\
  \hline
\end{tabularx}
\\
\newline
\\
\begin{tabularx}{\textwidth}{
  |p{\dimexpr.25\linewidth-2\tabcolsep-1.33333\arrayrulewidth}%
  |p{\dimexpr.75\linewidth-2\tabcolsep-1.33333\arrayrulewidth}|%
  }
  \hline
  \centering Identification &  SCAF-TC-28 \\
  \hline
  \centering Name & 變更密碼 \\
  \hline
  \centering Reference & SCAF-FR-11 \\
  \hline
  \centering Severity & 中 \\
  \hline
  \centering Instructions & 
  \makecell[l]{
    1. 點擊Navrbar中的Setting \\
    2. 填入舊密碼
    3. 新密碼 \\
    4. 點擊Save \\
    5. 登出系統 \\
    6. 輸入信箱、原始的密碼
  }\\
  \hline
  \centering Expected result & 系統顯示密碼錯誤 \\
  \hline
  \centering Cleanup & 無 \\
  \hline
\end{tabularx}
\\
\newline
\\
\begin{tabularx}{\textwidth}{
  |p{\dimexpr.25\linewidth-2\tabcolsep-1.33333\arrayrulewidth}%
  |p{\dimexpr.75\linewidth-2\tabcolsep-1.33333\arrayrulewidth}|%
  }
  \hline
  \centering Identification &  SCAF-TC-29 \\
  \hline
  \centering Name & 查看Q\&A \\
  \hline
  \centering Reference & SCAF-FR-12 \\
  \hline
  \centering Severity & 高 \\
  \hline
  \centering Instructions & 
  \makecell[l]{
    1. 點擊Navrbar中的Q\&A
  }\\
  \hline
  \centering Expected result & 顯示GitBook的內容 \\
  \hline
  \centering Cleanup & 無 \\
  \hline
\end{tabularx}
\\
\newline
\\
% \begin{tabularx}{\textwidth}{
%   |p{\dimexpr.25\linewidth-2\tabcolsep-1.33333\arrayrulewidth}%
%   |p{\dimexpr.75\linewidth-2\tabcolsep-1.33333\arrayrulewidth}|%
%   }
%   \hline
%   \centering Identification &  SCAF-TC-28 \\
%   \hline
%   \centering Name & Google 日曆授權 \\
%   \hline
%   \centering Reference & SCAF-FR-13 \\
%   \hline
%   \centering Severity & 1 \\
%   \hline
%   \centering Instructions & 
%   \makecell[l]{
%     1. 點擊Navrbar中的Setting \\
%     2. 點擊Google日曆旁的授權按鈕 \\
%     3. 按下授權 
%   }\\
%   \hline
%   \centering Expected result & 成功連結Google日曆 \\
%   \hline
%   \centering Cleanup & 無 \\
%   \hline
% \end{tabularx}
%\\
%\newline
%\\
% \begin{tabularx}{\textwidth}{
%   |p{\dimexpr.25\linewidth-2\tabcolsep-1.33333\arrayrulewidth}%
%   |p{\dimexpr.75\linewidth-2\tabcolsep-1.33333\arrayrulewidth}|%
%   }
%   \hline
%   \centering Identification &  SCAF-TC-29 \\
%   \hline
%   \centering Name & Google 日曆授權 \\
%   \hline
%   \centering Reference & SCAF-FR-13 \\
%   \hline
%   \centering Severity & 1 \\
%   \hline
%   \centering Instructions & 
%   \makecell[l]{
%     1. 點擊Navrbar中的Setting \\
%     2. 點擊Google日曆旁的授權按鈕 \\
%     3. 按下取消 
%   }\\
%   \hline
%   \centering Expected result & 未連結Google日曆 \\
%   \hline
%   \centering Cleanup & 無 \\
%   \hline
% \end{tabularx}
%\\
%\newline
%\\
\begin{tabularx}{\textwidth}{
  |p{\dimexpr.25\linewidth-2\tabcolsep-1.33333\arrayrulewidth}%
  |p{\dimexpr.75\linewidth-2\tabcolsep-1.33333\arrayrulewidth}|%
  }
  \hline
  \centering Identification &  SCAF-TC-30 \\
  \hline
  \centering Name & 變更專案名稱 \\
  \hline
  \centering Reference & SCAF-FR-14 \\
  \hline
  \centering Severity & 低 \\
  \hline
  \centering Instructions & 
  \makecell[l]{
    1. 點擊專案名稱下方的setting \\
    2. 點擊general頁面 \\
    3. 輸入新的專案名稱 \\
    4. 按下Update 
  }\\
  \hline
  \centering Expected result & 頁面、專案列表名稱皆更動成新的名稱 \\
  \hline
  \centering Cleanup & 無 \\
  \hline
\end{tabularx}
\\
\newline
\\
\begin{tabularx}{\textwidth}{
  |p{\dimexpr.25\linewidth-2\tabcolsep-1.33333\arrayrulewidth}%
  |p{\dimexpr.75\linewidth-2\tabcolsep-1.33333\arrayrulewidth}|%
  }
  \hline
  \centering Identification &  SCAF-TC-31 \\
  \hline
  \centering Name & 新增專案成員 \\
  \hline
  \centering Reference & SCAF-FR-14 \\
  \hline
  \centering Severity & 中 \\
  \hline
  \centering Instructions & 
  \makecell[l]{
    1. 專案擁有者點擊Project名稱下方的Setting \\
    2. 輸入已註冊過的使用者信箱 \\
    3. 接收者點擊通知中的專案邀請訊息 \\
    4. 接收者點擊確認加入專案 \\
    5. 專案擁有者確認member頁面
  }\\
  \hline
  \centering Expected result & 應增加加入者之訊息 \\
  \hline
  \centering Cleanup & 無 \\
  \hline
\end{tabularx}
\\
\newline
\\
\begin{tabularx}{\textwidth}{
  |p{\dimexpr.25\linewidth-2\tabcolsep-1.33333\arrayrulewidth}%
  |p{\dimexpr.75\linewidth-2\tabcolsep-1.33333\arrayrulewidth}|%
  }
  \hline
  \centering Identification &  SCAF-TC-32 \\
  \hline
  \centering Name & 剔除專案成員 \\
  \hline
  \centering Reference & SCAF-FR-14 \\
  \hline
  \centering Severity & 低 \\
  \hline
  \centering Instructions & 
  \makecell[l]{
    1. 專案擁有者點擊Project名稱下方的Setting \\
    2. 點擊任一位專案成員旁的X
    3. 被剔除人員確認專案列表
  }\\
  \hline
  \centering Expected result & 此專案應消失在專案列表中 \\
  \hline
  \centering Cleanup & 無 \\
  \hline
\end{tabularx}
\\
\newline
\\
\begin{tabularx}{\textwidth}{
  |p{\dimexpr.25\linewidth-2\tabcolsep-1.33333\arrayrulewidth}%
  |p{\dimexpr.75\linewidth-2\tabcolsep-1.33333\arrayrulewidth}|%
  }
  \hline
  \centering Identification &  SCAF-TC-33 \\
  \hline
  \centering Name & 專案初始化測試 \\
  \hline
  \centering Reference & SCAF-FR-06 \\
  \hline
  \centering Severity & 低 \\
  \hline
  \centering Instructions & 
  \makecell[l]{
    1. 點擊 My project 到專案列表頁面 \\
    2. 點擊 + 號創建一個新的專案 \\
    3. 輸入重複的專案名稱 \\
    4. 按下 Create
  }\\
  \hline
  \centering Expected result & 系統提示專案名稱已存在,專案列表未新增專案 \\
  \hline
  \centering Cleanup & 無 \\
  \hline
\end{tabularx}
\\
\newline
\\
\begin{tabularx}{\textwidth}{
  |p{\dimexpr.25\linewidth-2\tabcolsep-1.33333\arrayrulewidth}%
  |p{\dimexpr.75\linewidth-2\tabcolsep-1.33333\arrayrulewidth}|%
  }
  \hline
  \centering Identification &  SCAF-TC-34 \\
  \hline
  \centering Name & 專案初始化測試 \\
  \hline
  \centering Reference & SCAF-FR-06 \\
  \hline
  \centering Severity & 低 \\
  \hline
  \centering Instructions & 
  \makecell[l]{
    1. 點擊 My project 到專案列表頁面 \\
    2. 點擊 + 號創建一個新的專案 \\
    3. 輸入含非法字元的專案名稱 
    (如:/ \textbackslash ? \% * : |  " < >) \\
    4. 按下 Create
  }\\
  \hline
  \centering Expected result & 系統提示專案名稱不能有特殊字元,專案列表未新增專案 \\
  \hline
  \centering Cleanup & 無 \\
  \hline
\end{tabularx}
\\
\newline
\\
\begin{tabularx}{\textwidth}{
  |p{\dimexpr.25\linewidth-2\tabcolsep-1.33333\arrayrulewidth}%
  |p{\dimexpr.75\linewidth-2\tabcolsep-1.33333\arrayrulewidth}|%
  }
  \hline
  \centering Identification &  SCAF-TC-35 \\
  \hline
  \centering Name & 忘記密碼測試 \\
  \hline
  \centering Reference & SCAF-FR-03 \\
  \hline
  \centering Severity & 低 \\
  \hline
  \centering Instructions & 
  \makecell[l]{
    1. 在登入頁面點選Forgot Password?進入重設密碼頁面 \\
    2. 不輸入信箱  \\
    3. 點擊Submit \\
  }\\
  \hline
  \centering Expected result & 系統顯示錯誤 \\
  \hline
  \centering Cleanup & 無 \\
  \hline
\end{tabularx}
\\
\newline
\\
\begin{tabularx}{\textwidth}{
  |p{\dimexpr.25\linewidth-2\tabcolsep-1.33333\arrayrulewidth}%
  |p{\dimexpr.75\linewidth-2\tabcolsep-1.33333\arrayrulewidth}|%
  }
  \hline
  \centering Identification &  SCAF-TC-36 \\
  \hline
  \centering Name & 專案初始化測試 \\
  \hline
  \centering Reference & SCAF-FR-06 \\
  \hline
  \centering Severity & 低 \\
  \hline
  \centering Instructions & 
  \makecell[l]{
    1. 點擊My project到專案列表頁面  \\
    2. 點擊+號創建一個新的專案  \\
    3. 不輸入名稱  \\
    4. 按下Create
  }\\
  \hline
  \centering Expected result & 系統顯示錯誤,專案列表未出現新專案 \\
  \hline
  \centering Cleanup & 無 \\
  \hline
\end{tabularx}
\\
\newline
\\
\begin{tabularx}{\textwidth}{
  |p{\dimexpr.25\linewidth-2\tabcolsep-1.33333\arrayrulewidth}%
  |p{\dimexpr.75\linewidth-2\tabcolsep-1.33333\arrayrulewidth}|%
  }
  \hline
  \centering Identification &  SCAF-TC-37 \\
  \hline
  \centering Name & 新增repository \\
  \hline
  \centering Reference & SCAF-FR-07 \\
  \hline
  \centering Severity & 低 \\
  \hline
  \centering Instructions & 
  \makecell[l]{
    1. 點擊專案列表隨意一個專案 \\
    2. 點擊專案名稱下方的repositories \\
    3. 點擊+號創建一個新的repository \\
    3. 不輸入repository 名稱、對應的github網址  \\
    4. 按下Create
  }\\
  \hline
  \centering Expected result & 系統顯示錯誤,並不新增repository \\
  \hline
  \centering Cleanup & 無 \\
  \hline
\end{tabularx}
\\
\newline
\\

\begin{tabularx}{\textwidth}{
  |p{\dimexpr.25\linewidth-2\tabcolsep-1.33333\arrayrulewidth}%
  |p{\dimexpr.75\linewidth-2\tabcolsep-1.33333\arrayrulewidth}|%
  }
  \hline
  \centering Identification &  SCAF-TCCLI-01 \\
  \hline
  \centering Name & 登入帳號測試 \\
  \hline
  \centering Reference & SCAF-FR-01 \\
  \hline
  \centering Severity & 1 \\
  \hline
  \centering Instructions & 
  \makecell[l]{
    1. 在終端機中輸入scaf signin  \\
    2. 輸入信箱1@test.gmail(已註冊帳號) \\
    3. 輸入密碼123456 \\
  }\\
  \hline
  \centering Expected result & 顯示Sigin Success \\
  \hline
  \centering Cleanup & 無 \\
  \hline
\end{tabularx}
\\
\newline
\\

\begin{tabularx}{\textwidth}{
  |p{\dimexpr.25\linewidth-2\tabcolsep-1.33333\arrayrulewidth}%
  |p{\dimexpr.75\linewidth-2\tabcolsep-1.33333\arrayrulewidth}|%
  }
  \hline
  \centering Identification &  SCAF-TCCLI-02 \\
  \hline
  \centering Name & 註冊帳號測試 \\
  \hline
  \centering Reference & SCAF-FR-02 \\
  \hline
  \centering Severity & 1 \\
  \hline
  \centering Instructions & 
  \makecell[l]{
    1. 在終端機中輸入scaf signup\\
    2. 輸入符合格式的信箱註冊 \\
    3. 輸入至少6碼的密碼 \\
    4. 在終端機中輸入scaf signin  \\
    6. 輸入剛註冊帳號 \\
    7. 輸入密碼 \\
  }\\
  \hline
  \centering Expected result & 顯示Sigin Success \\
  \hline
  \centering Cleanup & 無 \\
  \hline
\end{tabularx}
\\
\newline
\\

\begin{tabularx}{\textwidth}{
  |p{\dimexpr.25\linewidth-2\tabcolsep-1.33333\arrayrulewidth}%
  |p{\dimexpr.75\linewidth-2\tabcolsep-1.33333\arrayrulewidth}|%
  }
  \hline
  \centering Identification &  SCAF-TCCLI-03 \\
  \hline
  \centering Name & 忘記密碼測試 \\
  \hline
  \centering Reference & SCAF-FR-03 \\
  \hline
  \centering Severity & 1 \\
  \hline
  \centering Instructions & 
  \makecell[l]{
    1. 在終端機中輸入scaf forgotpas\\
    2. 輸入已註冊信箱
    3. 在信箱中確認Firebase發送的重設密碼信件 \\
    4. 輸入新的密碼 \\
    5. 在終端機中輸入scaf signin\\
    6. 輸入信箱 \\
    7. 輸入新的密碼 \\
  }\\
  \hline
  \centering Expected result & 顯示Sigin Success \\
  \hline
  \centering Cleanup & 無 \\
  \hline
\end{tabularx}
\\
\newline
\\

\begin{tabularx}{\textwidth}{
  |p{\dimexpr.25\linewidth-2\tabcolsep-1.33333\arrayrulewidth}%
  |p{\dimexpr.75\linewidth-2\tabcolsep-1.33333\arrayrulewidth}|%
  }
  \hline
  \centering Identification &  SCAF-TCCLI-04 \\
  \hline
  \centering Name & 共用專案測試 \\
  \hline
  \centering Reference & SCAF-FR-04 \\
  \hline
  \centering Severity & 1 \\
  \hline
  \centering Instructions & 
  \makecell[l]{
    1. 在終端機中輸入scaf project clone \\
    2. 輸入 project author
    3. 輸入 project name
    4. cd project name\\
    5. 輸入scaf repo list查看有哪些repo \\
    6. 另一位專案成員用網頁端刪除一項repo \\
    7. 在終端機中輸入scaf project pull \\ 
    8. 輸入scaf repo list查看有哪些repo \\
  }\\
  \hline
  \centering Expected result & 顯示除了刪除的其他repo \\
  \hline
  \centering Cleanup & 無 \\
  \hline
\end{tabularx}
\\
\newline
\\

\begin{tabularx}{\textwidth}{
  |p{\dimexpr.25\linewidth-2\tabcolsep-1.33333\arrayrulewidth}%
  |p{\dimexpr.75\linewidth-2\tabcolsep-1.33333\arrayrulewidth}|%
  }
  \hline
  \centering Identification &  SCAF-TCCLI-05 \\
  \hline
  \centering Name & 專案初始化測試 \\
  \hline
  \centering Reference & SCAF-FR-05 \& SCAF-FR-06 \\
  \hline
  \centering Severity & 1 \\
  \hline
  \centering Instructions & 
  \makecell[l]{
    1. 在終端機中輸入scaf project create \\
    2. 選擇開發模式 \\
    3. 選擇是否授權google日歷、要不要看板功能 \\
    4. 在終端機中輸入scaf project list [ --oneline ]  \\
    5. 輸入 useremail \\
  }\\
  \hline
  \centering Expected result & 剛剛選擇的專案資訊符合列表中的資訊 \\
  \hline
  \centering Cleanup & 無 \\
  \hline
\end{tabularx}
\\
\newline
\\


\begin{tabularx}{\textwidth}{
  |p{\dimexpr.25\linewidth-2\tabcolsep-1.33333\arrayrulewidth}%
  |p{\dimexpr.75\linewidth-2\tabcolsep-1.33333\arrayrulewidth}|%
  }
  \hline
  \centering Identification &  SCAF-TCCLI-06 \\
  \hline
  \centering Name & 建立repository \\
  \hline
  \centering Reference & SCAF-FR-07 \\
  \hline
  \centering Severity & 1 \\
  \hline
  \centering Instructions & 
  \makecell[l]{
    1. 在終端機中輸入scaf repo list查看原本的有哪寫 \\
    2. scaf repo add <repo name> <repo url>
    3. 輸入 repo name \\
    4. 輸入 repo url \\
    5. 輸入scaf repo list查看有哪些repo \\
  }\\
  \hline
  \centering Expected result & 增加剛剛創立的repo \\
  \hline
  \centering Cleanup & 無 \\
  \hline
\end{tabularx}
\\
\newline
\\

\begin{tabularx}{\textwidth}{
  |p{\dimexpr.25\linewidth-2\tabcolsep-1.33333\arrayrulewidth}%
  |p{\dimexpr.75\linewidth-2\tabcolsep-1.33333\arrayrulewidth}|%
  }
  \hline
  \centering Identification &  SCAF-TCCLI-07 \\
  \hline
  \centering Name & 設定個人資料 \\
  \hline
  \centering Reference & SCAF-FR-11 \\
  \hline
  \centering Severity & 1 \\
  \hline
  \centering Instructions & 
  \makecell[l]{
    1. 在終端機中輸入scaf config set  \\
    2. 選擇user(<category>) \\
    3. 輸入bio(<field>) \\
    4. 輸入自界 \\
    5. 輸入scaf user \\
    6. 輸入信箱  \\
  }\\
  \hline
  \centering Expected result & bio被更新 \\
  \hline
  \centering Cleanup & 無 \\
  \hline
\end{tabularx}
\\
\newline
\\

\section*{4. 測試結果與分析(Test Results and Analysis)}
\subsection*{4.1測試結果(Test Results)}
\begin{tabularx}{\textwidth}{
  |p{\dimexpr.2\linewidth-2\tabcolsep-1.33333\arrayrulewidth}%
  |p{\dimexpr.2\linewidth-2\tabcolsep-1.33333\arrayrulewidth}%
  |p{\dimexpr.6\linewidth-2\tabcolsep-1.33333\arrayrulewidth}|%
}
  \hline
  測試案例編號 & 測試結果(Pass/Fail) & 註解 \\ \hline
  SCAF-TC-01 & Pass &  \\ \hline
  SCAF-TC-02 & Pass &  \\ \hline
  SCAF-TC-03 & Pass &  \\ \hline
  SCAF-TC-04 & Pass &  \\ \hline
  SCAF-TC-05 & Pass &  \\ \hline
  SCAF-TC-06 & Pass &  \\ \hline
  SCAF-TC-07 & Fail &  邀請成功,但無是否確認邀請,強制成功,前端member未正確顯示成員 \\ \hline
  SCAF-TC-08 & Pass &  \\ \hline
  SCAF-TC-09 & Pass &  \\ \hline
  SCAF-TC-10 & Pass &  \\ \hline
  SCAF-TC-11 & Fail &  無舉例設定流程日期的方式 \\ \hline
  SCAF-TC-12 & Fail &  沒有README檔 \\ \hline
  SCAF-TC-13 & Pass &  \\ \hline
  SCAF-TC-14 & Pass &  \\ \hline
  SCAF-TC-15 & Fail &  沒有顯示此連結網址無效,請重新輸入 \\ \hline
  SCAF-TC-16 & Pass &  \\ \hline
  SCAF-TC-17 & Fail &  沒有預覽的功能 \\ \hline
  SCAF-TC-18 & Pass &  \\ \hline
  SCAF-TC-19 & Pass &  \\ \hline
  SCAF-TC-20 & Pass &  \\ \hline
  SCAF-TC-21 & Pass &  \\ \hline
  SCAF-TC-22 & Fail & 無法使用 \\ \hline
  SCAF-TC-23 & Fail & 無法使用 \\ \hline
  SCAF-TC-24 & Fail & 無法使用\\ \hline
  SCAF-TC-25 & Fail & 未出現專案邀請,強制加入 \\ \hline
  SCAF-TC-26 & Pass &  \\ \hline
  SCAF-TC-27 & Pass &  \\ \hline
\end{tabularx}
\\
\newline
\\

\begin{tabularx}{\textwidth}{
  |p{\dimexpr.2\linewidth-2\tabcolsep-1.33333\arrayrulewidth}%
  |p{\dimexpr.2\linewidth-2\tabcolsep-1.33333\arrayrulewidth}%
  |p{\dimexpr.6\linewidth-2\tabcolsep-1.33333\arrayrulewidth}|%
}
  \hline
  SCAF-TC-28 & Pass &  \\ \hline
  SCAF-TC-29 & Pass &  \\ \hline
  SCAF-TC-30 & Fail & 沒有改變專案名稱 \\ \hline
  SCAF-TC-31 & Fail & 無法使用 \\ \hline
  SCAF-TC-32 & Fail & 無法顯示正確人員名單 \\ \hline
  SCAF-TC-33 & Fail & 可新增重複名稱的專案 \\ \hline
  SCAF-TC-34 & Fail & 仍可輸入非法字元 \\ \hline
  SCAF-TC-35 & Fail & 顯示寄送成功 \\ \hline
  SCAF-TC-36 & Fail & 未新增專案,但未出現錯誤提示 \\ \hline
  SCAF-TC-37 & Fail & 仍可新增空的repository \\ \hline
  SCAF-TCCLI-01 & Pass &  \\ \hline
  SCAF-TCCLI-02 & Pass &  \\ \hline
  SCAF-TCCLI-03 & Pass &  \\ \hline
  SCAF-TCCLI-04 & Pass &  \\ \hline
  SCAF-TCCLI-05 & Pass &  \\ \hline
  SCAF-TCCLI-06 & Pass &  \\ \hline
  SCAF-TCCLI-07 & Pass &  \\ \hline
\end{tabularx}
\\
\newline
\\

\subsection*{4.2缺失報告(Defect Tracking)}
\begin{tabularx}{\textwidth}{ 
  |p{\dimexpr.1\linewidth-2\tabcolsep-1.33333\arrayrulewidth}%
  |p{\dimexpr.1\linewidth-2\tabcolsep-1.33333\arrayrulewidth}%
  |p{\dimexpr.2\linewidth-2\tabcolsep-1.33333\arrayrulewidth}%
  |p{\dimexpr.2\linewidth-2\tabcolsep-1.33333\arrayrulewidth}%
  |p{\dimexpr.1\linewidth-2\tabcolsep-1.33333\arrayrulewidth}%
  |p{\dimexpr.1\linewidth-2\tabcolsep-1.33333\arrayrulewidth}%
  |p{\dimexpr.2\linewidth-2\tabcolsep-1.33333\arrayrulewidth}|%
}
  \hline
  缺失標號 & 缺失嚴重性 & 缺失說明 & 測試案例編號 & 缺失負責人 & 修復狀態 & 修復說明\\
  \hline
  001 & 中 & 邀請成功,但無是否確認邀請,強制成功,前端 member 未正確顯示成員 & SCAF-TC-07 & 唐劭賢 &ongoing & \\
  \hline
  002 & 低 & 無法設定流程日期 & SCAF-TC-11 & 余威霆 & ongoing & \\
  \hline
  003 & 中 & 專案內文件沒有README檔 & SCAF-TC-12 & 唐劭賢 & ongoing & \\
  \hline
  004 & 低 & 沒有顯示無效仍可新增 & SCAF-TC-15 & 唐劭賢 & ongoing & \\
  \hline
  005 & 中 & 無法瀏覽已存在的檔案 & SCAF-TC-17 & 林佳何 & ongoing & \\
  \hline
  006 & 高 & 無法使用看板 & SCAF-TC-22 & 唐劭賢 & closed & 改用箭頭做移動 \\
  \hline
  007 & 低 & 無法使用看板 & SCAF-TC-23 & 唐劭賢 & closed & 改用箭頭做移動 \\
  \hline
  008 & 低 & 無法更改文件內容 & SCAF-TC-24 & 余威霆 & ongoing & \\
  \hline
  009 & 中 & 未顯示邀請,強制加入 & SCAF-TC-25 & 簡蔚驊 & ongoing & \\
  \hline
\end{tabularx}
\\
\newline
\\
\begin{tabularx}{\textwidth}{
  |p{\dimexpr.1\linewidth-2\tabcolsep-1.33333\arrayrulewidth}%
  |p{\dimexpr.1\linewidth-2\tabcolsep-1.33333\arrayrulewidth}%
  |p{\dimexpr.2\linewidth-2\tabcolsep-1.33333\arrayrulewidth}%
  |p{\dimexpr.2\linewidth-2\tabcolsep-1.33333\arrayrulewidth}%
  |p{\dimexpr.1\linewidth-2\tabcolsep-1.33333\arrayrulewidth}%
  |p{\dimexpr.1\linewidth-2\tabcolsep-1.33333\arrayrulewidth}%
  |p{\dimexpr.2\linewidth-2\tabcolsep-1.33333\arrayrulewidth}|%
}
  \hline
  010 & 低 & 無法變更專案名稱 & SCAF-TC-30 & 簡蔚驊 & ongoing & \\
  \hline
  011 & 中 & 無法確認成員名單 & SCAF-TC-31 & 林佳何 & ongoing & \\
  \hline
  012 & 低 & 無法確認成員名單 & SCAF-TC-32 & 林佳何 & ongoing & \\
  \hline
  013 & 低 & 可新增重複名稱的專案 & SCAF-TC-34 & 余威霆 & closed & 已加入判斷專案名稱是否重複 \\
  \hline
  014 & 低 & 可新增含非法字元名稱的專案 & SCAF-TC-35 & 唐劭賢  & closed & 已加入判斷非法字元,若有非法字元則不能建立 \\
  \hline
  015 & 低 & 顯示寄送成功 & SCAF-TC-36 & 簡蔚驊 & closed & 加入判斷欄位不得為空 \\
  \hline
  016 & 低 & 未新增專案,但未出現錯誤提示 & SCAF-TC-37 & 林佳何 & closed & 新增錯誤提示 \\
  \hline
  017 & 低 & 仍可新增空的repository & SCAF-TC-38 & 余威霆 & closed & 加入判斷欄位不得為空 \\
  \hline
\end{tabularx}
\\
\newline
\\
\section*{5. JMeter效能測試}
\graphicspath{ {./images} }
(SCAF-NFP-01 系統人數同時負荷 1000 人)\\
(SCAF-NFP-02 回應時間在 1 秒內)\\
設定執行序數量1000,系統反應平均為55ms,沒有錯誤\\
\includegraphics[scale=0.35]{1000_1.png}
\includegraphics[scale=0.35]{1000_2.png}
\\
\newline
\\
設定執行序數量2000,系統反應平均為61ms,沒有錯誤 \\
\includegraphics[scale=0.35]{2000_1.png}
\includegraphics[scale=0.35]{2000_2.png}
\\
\newline
\\
設定執行序數量3000,出現錯誤 \\
\includegraphics[scale=0.35]{3000.png}
\\
\newline
\\
\section*{6. 追溯表(Traceability Matrix)}
\begin{tabularx}{\textwidth}{ 
  |p{\dimexpr.4\linewidth-2\tabcolsep-1.33333\arrayrulewidth}%
  |p{\dimexpr.4\linewidth-2\tabcolsep-1.33333\arrayrulewidth}%
  |p{\dimexpr.2\linewidth-2\tabcolsep-1.33333\arrayrulewidth}|%
}
  \hline
  Req. No. & Test Case \#Verification & Verification \\
  \hline
  SCAF-FR-01 & SCAF-TC-01 & Verified \\
  \hline
  SCAF-FR-01 & SCAF-TC-02 & Verified \\
  \hline
  SCAF-FR-02 & SCAF-TC-03 & Verified \\
  \hline
  SCAF-FR-02 & SCAF-TC-04 & Verified \\
  \hline
  SCAF-FR-03 & SCAF-TC-05 & Verified \\
  \hline
  SCAF-FR-03 & SCAF-TC-06 & Verified \\
  \hline
  SCAF-FR-04 & SCAF-TC-07 & Verified \\
  \hline
  SCAF-FR-04 & SCAF-TC-08 & Verified \\
  \hline
  SCAF-FR-04 & SCAF-TC-09 & Verified \\
  \hline
  SCAF-FR-04 & SCAF-TC-10 & Verified \\
  \hline
  SCAF-FR-05 & SCAF-TC-11 & Verified \\
  \hline
  SCAF-FR-06 & SCAF-TC-12 & Verified \\
  \hline
  SCAF-FR-06 & SCAF-TC-13 & Verified \\
  \hline
  SCAF-FR-07 & SCAF-TC-14 & Verified \\
  \hline
  SCAF-FR-07 & SCAF-TC-15 & Verified \\
  \hline
  SCAF-FR-07 & SCAF-TC-16 & Verified \\
  \hline
  SCAF-FR-08 & SCAF-TC-17 & Verified \\
  \hline
  SCAF-FR-08 & SCAF-TC-18 & Verified \\
  \hline
  SCAF-FR-08 & SCAF-TC-19 & Verified \\
  \hline
  SCAF-FR-09 & SCAF-TC-20 & Verified \\
  \hline
  SCAF-FR-09 & SCAF-TC-21 & Verified \\
  \hline
  SCAF-FR-09 & SCAF-TC-22 & Verified \\
  \hline
  SCAF-FR-09 & SCAF-TC-23 & Verified \\
  \hline
  SCAF-FR-10 & SCAF-TC-24 & Verified \\
  \hline
  SCAF-FR-10 & SCAF-TC-25 & Verified \\
  \hline
  SCAF-FR-11 & SCAF-TC-26 & Verified \\
  \hline
  SCAF-FR-11 & SCAF-TC-27 & Verified \\
  \hline
  SCAF-FR-11 & SCAF-TC-28 & Verified \\
  \hline
  SCAF-FR-12 & SCAF-TC-29 & Verified \\
  \hline
  SCAF-FR-14 & SCAF-TC-30 & Verified \\
  \hline
  SCAF-FR-14 & SCAF-TC-31 & Verified \\
  \hline
  SCAF-FR-14 & SCAF-TC-32 & Verified \\
  \hline
  SCAF-FR-06 & SCAF-TC-33 & Verified \\
  \hline
  SCAF-FR-06 & SCAF-TC-34 & Verified \\
  \hline
  SCAF-FR-03 & SCAF-TC-35 & Verified \\
  \hline
\end{tabularx}
\newpage
\begin{tabularx}{\textwidth}{ 
  |p{\dimexpr.4\linewidth-2\tabcolsep-1.33333\arrayrulewidth}%
  |p{\dimexpr.4\linewidth-2\tabcolsep-1.33333\arrayrulewidth}%
  |p{\dimexpr.2\linewidth-2\tabcolsep-1.33333\arrayrulewidth}|%
}
  \hline
  SCAF-FR-06 & SCAF-TC-36 & Verified \\
  \hline
  SCAF-FR-07 & SCAF-TC-37 & Verified \\
  \hline
  SCAF-FR-01 & SCAF-TCCLI-01 & Verified \\
  \hline
  SCAF-FR-02 & SCAF-TCCLI-02 & Verified \\
  \hline
  SCAF-FR-03 & SCAF-TCCLI-03 & Verified \\
  \hline
  SCAF-FR-04 & SCAF-TCCLI-04 & Verified \\
  \hline
  SCAF-FR-05 、SCAF-FR-06 & SCAF-TCCLI-05 & Verified \\
  \hline
  SCAF-FR-07 & SCAF-TCCLI-06 & Verified \\
  \hline
\end{tabularx}
\end{document}




