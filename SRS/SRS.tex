\documentclass{report}
\usepackage[margin=2cm, bottom=1.5cm]{geometry}
\usepackage{type1cm}
\usepackage{amssymb}
\usepackage[fleqn]{amsmath}
\usepackage{tikz}
\usepackage{multicol}
\usepackage{makecell}
\usepackage{tabularx}
\usepackage[shortlabels]{enumitem}
\setlength{\columnsep}{1pt}
\setlist[enumerate]{nosep}

\makeatletter
\newenvironment{myalign*}{\ifvmode\else\hfil\null\linebreak\fi
  \hspace*{-\leftmargin}\minipage\textwidth
  \setlength{\abovedisplayskip}{0pt}%
  \setlength{\abovedisplayshortskip}{\abovedisplayskip}%
  \start@align\@ne\st@rredtrue\m@ne}%
{\endalign\endminipage\linebreak}

\usepackage{xeCJK}
\setCJKmainfont{Noto Sans TC}

\begin{document}
\title{%
  \fontsize{40}{60}\selectfont
  SCAF  \\ % 
  \vspace*{2cm}%
  \fontsize{24}{30}\selectfont
  需求規格文件
}

\author{
  \fontsize{18}{28}\selectfont
  \begin{tabularx}{0.9\textwidth}{
    |p{\dimexpr.25\linewidth-6\tabcolsep-2\arrayrulewidth}%
    |p{\dimexpr.75\linewidth-6\tabcolsep-2\arrayrulewidth}|%
  }
    \hline
    \centering 專案名稱 & SCAF 開發輔助工具 \\
    \hline
    \centering 撰寫日期 & 2022 / 10 / 19 \\
    \hline
    \centering 發展者 & 簡蔚驊 \! 鄧暐宣 \! 余威霆 \! 林佳何 \! 唐劭賢 \\
    \hline
  \end{tabularx}
}
\date{}
\usetikzlibrary{automata, positioning, arrows}
\maketitle
\tikzset{every state, accepting/.style={double distance=2pt}}

\fontsize{12}{18}\selectfont
\section*{0. 接受準則(Acceptance Criteria of this document)}
\begin{itemize}
  \item Clearly and properly stated (需求需清楚且適當的陳述)
  \item Complete (需求需完整)
  \item Consistent with each other (需求之間需維持一致性)
  \item Uniquely identified (每項需求有明確之識別)
  \item Appropriate to implement (需求需可被實作)
  \item Verifiable (需求需可被驗證)
\end{itemize}

\section*{1. 系統概述(System Description)}
目標: 以簡化的方式呈現 SCAF 的功能,並且提供使用者一個簡單的操作介面,讓使用者能夠輕鬆的使用 SCAF 的功能。\\
特色: 簡單、易用、快速。\\
實作方案:
  \begin{itemize}
    \item 使用 Adobe Illustrator 設計 UI 介面
    \item 使用 Vue 技術實現前端頁面
    \item 使用 Go 設計 CLI 工具
    \item 使用 Go 設計後端 API
  \end{itemize}
\includegraphics[width=\textwidth]{assets/system.png}

\section*{2. 使用者故事地圖(User Story Map)}

% \subsection*{}
% \fontsize{12}{20}\selectfont
% \begin{tabularx}{\textwidth}{|X|}
%   \hline
%   代號:\\ \hline
%   故事: \\ \hline
%   註記: \\ \hline
%   測試方案: \\ \hline
% \end{tabularx}

\subsection*{}
\fontsize{12}{20}\selectfont
\begin{tabularx}{\textwidth}{|X|}
  \hline
  代號:SCAF-US-01 登入/註冊 \\ \hline
  故事:作為一個使用者
  我可以註冊帳號或連動第三方應用程式,讓我可以登入這個系統。 \\ \hline
  註記:
  \begin{enumerate}
    \item 使用者須先註冊才能使用
    \item 使用者也可以使用 Google 或 Github 登入
  \end{enumerate} \\ \hline
  測試方案:
  \begin{enumerate}
    \item 若註冊時輸入不存在的 email,確認是否有 email 不存在的錯誤訊息
    \item 若 email 未註冊登入,確認是否有 email 未註冊的錯誤訊息。
    \item 若密碼輸入錯誤,確認是否有密碼錯誤的錯誤訊息
    \item 辦完帳號以後,輸入電子郵件以及密碼看是否能登入
    \item 使用 Google 帳號登入,確認是否能夠連動
    \item 使用 Github 帳號登入,確認是否能夠連動
  \end{enumerate} \\ \hline
\end{tabularx}

\subsection*{}
\fontsize{12}{20}\selectfont
\begin{tabularx}{\textwidth}{|X|}
  \hline
  代號:SCAF-US-02 修改個人資料 \\
  \hline
  故事:作為一個使用者,我要能夠修改自己的個人資料,像是使用者名稱太中二
想修改、密碼取得太簡單想改困難一點、想取消第三方的應用程式連動。 \\
  \hline
  註記:
  \begin{enumerate}
    \item 使用者可以修改名稱
    \item 使用者可以修改密碼
    \item 使用者可以新增或取消第三方應用程式連動
    \item 系統提醒使用者確認變更
  \end{enumerate} \\
  \hline
  測試方案:
  \begin{enumerate}
    \item 修改名稱或密碼時,未輸入字元送出,系統顯示不能為空的錯誤訊息
    \item 修改使用名稱後,進入系統看是否為新名稱
    \item 修改密碼後,登出系統再用新密碼登入,確認是否能成功
    \item 新增或取消第三方應用程式連動後,重新登入,查看是否可以正常登入
  \end{enumerate} \\
  \hline
\end{tabularx}

\subsection*{}
\fontsize{12}{20}\selectfont
\begin{tabularx}{\textwidth}{|X|}
  \hline
  代號:SCAF-US-03 執行命令 \\
  \hline
  故事:作為一個使用者,我能夠進入CLI介面並在此輸入我想要的指令,像是使用init來初始化專案。 \\
  \hline
  註記:
  \begin{enumerate}
    \item 指令有固定的格式規範
    \item 有help指令可以查看指令的說明
  \end{enumerate} \\
  \hline
  測試方案:
  \begin{enumerate}
    \item 輸入指令時,輸入錯誤的指令,確認是否有錯誤訊息
    \item 輸入指令時,輸入正確的指令,確認是否有正確執行
    \item 輸入指令時,輸入 help,確認是否有列出所有指令
    \item 輸入指令時,輸入 help 指令,確認是否有列出該指令的說明
  \end{enumerate} \\
  \hline
\end{tabularx}

\subsection*{}
\fontsize{12}{20}\selectfont
\begin{tabularx}{\textwidth}{|X|}
  \hline
  代號:SCAF-US-04 添加開發工具 \\
  \hline
  故事:作為一個使用者,我能夠在專案開始時(或是後續開發階段), 選擇系統的選項,並查看系統所推薦的開發工具,或者自行添加開發工具。 \\
  \hline
  註記:
  \begin{enumerate}
    \item 系統會有幾個固定選項,使用者可以選擇其中一個
    \item 若不滿意系統推薦的選項,使用者可以自行添加開發工具
    \item 後續的開發階段,使用者可以選擇更換開發工具
  \end{enumerate} \\
  \hline
  測試方案:
  \begin{enumerate}
    \item 在專案開始時,使用者可以選擇其中一個固定選項
    \item 添加完開發工具後,進入專案,確認是否有正確添加
    \item 移除開發工具後,進入專案,確認是否有正確移除
  \end{enumerate} \\
  \hline
\end{tabularx}

\subsection*{}
\fontsize{12}{20}\selectfont
\begin{tabularx}{\textwidth}{|X|}
  \hline
  代號:SCAF-US-05 開發流程Q\&A \\
  \hline
  故事:作為一個使用者,當我對於開發流程有所疑惑時,可以打開開發流程Q\&A文件,查看自己的疑惑是否在常見問題中,或是需要自行上網查詢。 \\
  \hline
  註記:
  \begin{enumerate}
    \item 文件會條列出開發流程階段的常見問題
    \item 提供關鍵字搜尋功能
  \end{enumerate} \\
  \hline
  測試方案:
  \begin{enumerate}
    \item 點開文件,確認是否有列出開發流程階段的常見問題
    \item 輸入不存在的關鍵字,確認是否有顯示沒有找到相關問題
  \end{enumerate} \\
  \hline
\end{tabularx}

\subsection*{}
\fontsize{12}{20}\selectfont
\begin{tabularx}{\textwidth}{|X|}
  \hline
  代號:SCAF-US-06 需求分析 以文件形式儲存分析結果 \\
  \hline
  故事:作為一個使用者,當我在撰寫需求文件時,我能夠將我的需求分析文件處存在此軟體中,並在我需要時叫出它做修改。 \\
  \hline
  註記:
  \begin{enumerate}
    \item 將使用hackmd撰寫的需求分析文件,儲存在此軟體中
    \item 需求分析文件會有一個獨立的頁面
  \end{enumerate} \\
  \hline
  測試方案:
  \begin{enumerate}
    \item 新增一個需求分析文件
    \item 編輯需求分析文件
    \item 刪除需求分析文件
    \item 在其他開發階段開啟需求分析文件
  \end{enumerate} \\
  \hline
\end{tabularx}

\subsection*{}
\fontsize{12}{20}\selectfont
\begin{tabularx}{\textwidth}{|X|}
  \hline
  代號:SCAF-US-07 設計階段 以文件形式儲存分析結果 \\
  \hline
  故事:作為一個使用者,當我在撰寫設計文件時,我能夠將我的需求分析文件處存在此軟體中,並可以加入線框圖之類的img,並在我需要時叫出它做修改。 \\
  \hline
  註記:
  \begin{enumerate}
    \item 將使用hackmd撰寫的需求分析文件,儲存在此軟體中
    \item 設計文件會有一個獨立的頁面
    \item 可以加入圖片
  \end{enumerate} \\
  \hline
  測試方案:
  \begin{enumerate}
    \item 新增一個設計文件
    \item 編輯需求設計文件
    \item 刪除需求設計文件
    \item 在其他開發階段開啟設計文件
  \end{enumerate} \\
  \hline
\end{tabularx}

\subsection*{}
\fontsize{12}{20}\selectfont
\begin{tabularx}{\textwidth}{|X|}
  \hline
  代號:SCAF-US-08 開發環境設定文件 \\
  \hline
  故事:作為一個使用者,我可以將自己使用的程式語言版本,或是測試使用的軟體,寫入到開發環境設定文件中,或是查看他人使用的是甚麼環境,來確保整個專案的環境統一。 \\
  \hline
  註記:
  \begin{enumerate}
    \item 將使用hackmd撰寫的需求分析文件,儲存在此軟體中
    \item 更改文件後,可以讓更改者決定是否發文件通知所有人
    \item 開發環境設定文件會有一個獨立的頁面
  \end{enumerate} \\
  \hline
  測試方案:
  \begin{enumerate}
    \item 新增一個開發環境設定文件
    \item 編輯開發環境設定文件
    \item 刪除開發環境設定文件
    \item 更改文件後,選擇發送通知,確認每一個人都有收到通知
  \end{enumerate} \\
  \hline
\end{tabularx}


\subsection*{3. 操作概念(Operation
 Concept)}

\begin{enumerate}[label=(\Alph*)]
  \item 會員註冊: 小唐進入了這個 SCAF 平台,輸入名稱、信箱、密碼,註冊了一個帳號。
  \item 會員登入: 小唐辦好帳號後,馬上試著登入平台來開始自己的軟體開發流程。
  \item 忘記密碼: 登入時卻發現,輸入好多次密碼都是錯的,只好重設密碼,輸入信箱之後,查看了自己的信箱,並點擊 Link 去重新設定密碼。
\end{enumerate}

\subsection*{4. 功能需求(Feature Requirements)}

\begin{tabularx}{\textwidth}{
  |p{\dimexpr.2\linewidth-2\tabcolsep-1.33333\arrayrulewidth}%
  |p{\dimexpr.2\linewidth-2\tabcolsep-1.33333\arrayrulewidth}%
  |p{\dimexpr.6\linewidth-2\tabcolsep-1.33333\arrayrulewidth}|%
}
  \hline
  功能編號 & 功能名稱 & 功能說明 \\ \hline
  SCAF-FR-01 & 登入 & 使用者可以用 email 跟密碼進行登入,或是使用 Google 和 Github \\ \hline
  SCAF-FR-02 & 註冊 & 使用者可以用 email 註冊新帳戶。 \\ \hline
  SCAF-FR-03 & 忘記密碼 & 使用者可以發送驗證碼至 email 並重設密碼。 \\ \hline
  SCAF-FR-04 & 共用專案 & 使用者可以分享連結讓多人參與共同編輯。 \\ \hline
  SCAF-FR-05 & 開發流程模式 & 使用者可以選擇自己想要的設計流程。 \\ \hline
  SCAF-FR-06 & 專案初始化 & 使用者可以生成前後端基本的文件,添加外部的插件。 \\ \hline
  SCAF-FR-07 & 需求分析 & 使用者可以填入與客戶商討後的需求。 \\ \hline
  SCAF-FR-08 & 設計文件 & 使用者可以填入 API 形式和資料庫規範。 \\ \hline
  SCAF-FR-09 & 設定開發環境 & 使用者可以在 Docker 中實作。 \\ \hline
  SCAF-FR-10 & 命名規範 & 使用者可以在命名錯誤時修正。 \\ \hline
  SCAF-FR-11 & 風格規範 & 方便閱讀他人 code。 \\ \hline
  SCAF-FR-12 & 版本控管 & 使用者可以利用 git 各項功能。 \\ \hline
  SCAF-FR-13 & 工作計畫排程 & 使用者可以利用看板、行事曆、留言板,來觀察專案的排程。 \\ \hline
  SCAF-FR-14 & 整合其他工具 & 使用者可以加入自己想要的模組或是平台。 \\ \hline
  SCAF-FR-15 & 自動化測試 & 使用者可以寫相關測試程式碼,commit 時能夠自動測試。 \\ \hline
  SCAF-FR-16 & 自動化部屬 & 使用者可以用 CI/CD 工具 \\ \hline
  SCAF-FR-17 & 產生規範文件 & 依據設計文件產生說明文件 \\ \hline
  SCAF-FR-18 & 問答 & 使用者可以問題提交上來,系統引導使用者排解問題 \\ \hline
\end{tabularx}

\subsection*{5. 非功能需求(Non-functional Requirements)}

\begin{tabularx}{\textwidth}{
  |p{\dimexpr.3\linewidth-2\tabcolsep-2\arrayrulewidth}%
  |p{\dimexpr.7\linewidth-2\tabcolsep-2\arrayrulewidth}|%
}
  \hline
  功能編號 &  功能說明 \\ \hline
  SCAF-NFR-01 & 使用 yaml 格式儲存文件 \\ \hline
\end{tabularx}

\subsection*{6. 使用者介面分析(User Interface Analysis)}
\begin{enumerate}
  \item 登入頁面:
    \begin{enumerate}
      \item 輸入 email 及 password 登入(需先註冊)
      \item 使用 Google 帳號或 Github 帳號直接登入
      \item 註冊
      \item 忘記密碼
    \end{enumerate}
  \item 註冊頁面: 填入使用者名稱、信箱、密碼
  \item 忘記密碼: 輸入信箱後,系統會發送連結到信箱中,可以點擊連結重設密碼
\end{enumerate}

\end{document}