\documentclass{report}
\usepackage[margin=2cm, bottom=1.5cm]{geometry}
\usepackage{type1cm}
\usepackage{amssymb}
\usepackage[fleqn]{amsmath}
\usepackage{tikz}
\usepackage{multicol}
\usepackage{makecell}
\usepackage{tabularx}
\usepackage[shortlabels]{enumitem}
\setlength{\columnsep}{1pt}
\setlist[enumerate]{nosep}

\makeatletter
\newenvironment{myalign*}{\ifvmode\else\hfil\null\linebreak\fi
  \hspace*{-\leftmargin}\minipage\textwidth
  \setlength{\abovedisplayskip}{0pt}%
  \setlength{\abovedisplayshortskip}{\abovedisplayskip}%
  \start@align\@ne\st@rredtrue\m@ne}%
{\endalign\endminipage\linebreak}

\usepackage{xeCJK}
\setCJKmainfont{Noto Sans TC}

\begin{document}
\title{%
  \fontsize{40}{60}\selectfont
  SCAF  \\ % 
  \vspace*{2cm}%
  \fontsize{24}{30}\selectfont
  需求規格文件
}

\author{
  \fontsize{18}{28}\selectfont
  \begin{tabularx}{0.9\textwidth}{
    |p{\dimexpr.25\linewidth-6\tabcolsep-2\arrayrulewidth}%
    |p{\dimexpr.75\linewidth-6\tabcolsep-2\arrayrulewidth}|%
  }
    \hline
    \centering 專案名稱 & SCAF 開發輔助工具 \\
    \hline
    \centering 撰寫日期 & 2022 / 10 / 19 \\
    \hline
    \centering 發展者 & 簡蔚驊 \! 鄧暐宣 \! 余威霆 \! 林佳何 \! 唐劭賢 \\
    \hline
  \end{tabularx}
}
\date{}
\usetikzlibrary{automata, positioning, arrows}
\maketitle
\tikzset{every state, accepting/.style={double distance=2pt}}

\fontsize{12}{18}\selectfont
\section*{0. 接受準則(Acceptance Criteria of this document)}
\begin{itemize}
  \item Clearly and properly stated (需求需清楚且適當的陳述)
  \item Complete (需求需完整)
  \item Consistent with each other (需求之間需維持一致性)
  \item Uniquely identified (每項需求有明確之識別)
  \item Appropriate to implement (需求需可被實作)
  \item Verifiable (需求需可被驗證)
\end{itemize}

\section*{1. 系統概述(System Description)}
目標: 以簡化的方式呈現 SCAF (Software Co-working Assistance Framework)的功能,
並且提供使用者一個簡單的操作介面,讓使用者能夠輕鬆的使用 SCAF 的功能。\\
特色: 簡單、易用、快速。\\
實作方案:
  \begin{itemize}
    \item 使用 Adobe XD (Adobe Experience Design)設計 UI 介面
    \item 使用 Vue 技術實現前端頁面
    \item 使用 Go 設計 CLI 工具
    \item 使用 Go 設計後端 API
  \end{itemize}
\includegraphics[width=\textwidth]{assets/system.png}

\section*{2. 使用者故事地圖(User Story Map)}

\includegraphics[width=\textwidth]{assets/User_Story_Map.png}

% \subsection*{}
% \fontsize{12}{20}\selectfont
% \begin{tabularx}{\textwidth}{|X|}
%   \hline
%   代號:\\ \hline
%   故事: \\ \hline
%   註記: \\ \hline
%   測試方案: \\ \hline
% \end{tabularx}

\subsection*{}
\fontsize{12}{20}\selectfont
\begin{tabularx}{\textwidth}{|X|}
  \hline
  代號:SCAF-US-01 登入\\ \hline
  故事:作為一個使用者我可以輸入已註冊的帳號登入這個系統。 \\ \hline
  註記:
  \begin{enumerate}
    \item 登入要輸入電子郵件和密碼
    \item 登入失敗會出現錯誤訊息
    \item 使用者可以使用第三方登入 
  \end{enumerate} \\ \hline
  測試方案:
  \begin{enumerate}
    \item 輸入未註冊的電子郵件與密碼,確認是否失敗與顯示錯誤訊息
    \item 辦完帳號以後,輸入電子郵件以及密碼看是否能登入
    \item 使用第三方登入,確認是否能夠連動
  \end{enumerate} \\ \hline
\end{tabularx}

\subsection*{}
\fontsize{12}{20}\selectfont
\begin{tabularx}{\textwidth}{|X|}
  \hline
  代號:SCAF-US-02 註冊 \\ \hline
  故事:作為一個使用者我可以註冊新帳戶或連動第三方應用程式,讓我可以登入這個系統。 \\ \hline
  註記:
  \begin{enumerate}
    \item 註冊時需輸入電子郵件、用戶名、密碼
    \item 註冊電子郵件為無效的電子郵件,出現錯誤訊息
  \end{enumerate} \\ \hline
  測試方案:
  \begin{enumerate}
    \item 若註冊時輸入無效的 email,確認是否有錯誤訊息
  \end{enumerate} \\ \hline
\end{tabularx}

\subsection*{}
\fontsize{12}{20}\selectfont
\begin{tabularx}{\textwidth}{|X|}
  \hline
  代號:SCAF-US-03 修改個人資料 \\
  \hline
  故事:作為一個使用者,我要能夠修改自己的個人資料,像是使用者名稱太中二
想修改、密碼取得太簡單想改困難一點、想取消第三方的應用程式連動。 \\
  \hline
  註記:
  \begin{enumerate}
    \item 使用者可以修改名稱,並自動登出
    \item 使用者可以修改密碼,並自動登出
    \item 使用者可以新增或取消第三方應用程式連動,並自動登出
    \item 系統提醒使用者確認變更
    \item 修改名稱或密碼時,未輸入字元送出,出現錯誤訊息
  \end{enumerate} \\
  \hline
  測試方案:
  \begin{enumerate}
    \item 修改名稱或密碼時,未輸入字元送出,確認是否有錯誤訊息
    \item 修改使用名稱後,重新登入,進入系統看是否為新名稱
    \item 新增或取消第三方應用程式連動後,重新登入,查看是否可以正常登入
    \item 修改密碼後,再用新密碼登入,確認是否能成功登入
    \item 修改名稱後,確認是否自動登出
    \item 修改密碼後,確認是否自動登出
    \item 新增或取消第三方應用程式連動後,確認是否自動登出
  \end{enumerate} \\
  \hline
\end{tabularx}

\subsection*{}
\fontsize{12}{20}\selectfont
\begin{tabularx}{\textwidth}{|X|}
  \hline
  代號:SCAF-US-04 執行命令 \\
  \hline
  故事:作為一個使用者,我能夠進入CLI介面並在此輸入我想要的指令,像是使用init來初始化專案。 \\
  \hline
  註記:
  \begin{enumerate}
    \item 指令有固定的格式規範
    \item 有help指令可以查看指令的說明
    \item 所有網站的功能都可以透過指令來執行
    \item 輸入指令時,輸入錯誤的指令,會出現錯誤訊息
  \end{enumerate} \\
  \hline
  測試方案:
  \begin{enumerate}
    \item 輸入指令時,輸入錯誤的指令,確認是否有錯誤訊息
    \item 輸入指令時,輸入正確的指令,確認是否有正確執行
    \item 輸入指令時,輸入 help,確認是否有列出所有指令
    \item 輸入指令時,輸入 help 指令,確認是否有列出該指令的說明
  \end{enumerate} \\
  \hline
\end{tabularx}

\subsection*{}
\fontsize{12}{20}\selectfont
\begin{tabularx}{\textwidth}{|X|}
  \hline
  代號:SCAF-US-05 添加開發工具 \\
  \hline
  故事:作為一個使用者,我能夠在專案開始時(或是後續開發階段), 選擇系統的選項,並查看系統所推薦的開發工具,或者自行添加開發工具。 \\
  \hline
  註記:
  \begin{enumerate}
    \item 系統會有幾個固定選項,使用者可以選擇其中一個
    \item 若不滿意系統推薦的選項,使用者可以自行添加開發工具
    \item 後續的開發階段,使用者可以選擇更換開發工具
  \end{enumerate} \\
  \hline
  測試方案:
  \begin{enumerate}
    \item 在專案開始時,使用者可以選擇其中一個固定選項
    \item 添加完開發工具後,進入專案,確認是否有正確添加
    \item 移除開發工具後,進入專案,確認是否有正確移除
  \end{enumerate} \\
  \hline
\end{tabularx}

\subsection*{}
\fontsize{12}{20}\selectfont
\begin{tabularx}{\textwidth}{|X|}
  \hline
  代號:SCAF-US-06 開發、測試、發布\&部屬 Q\&A \\
  \hline
  故事:作為一個使用者,當我對於開發、測試、發布\&部屬流程有所疑惑時,可以打開開發流程Q\&A文件,查看自己的疑惑是否在常見問題中。 \\
  \hline
  註記:
  \begin{enumerate}
    \item 文件會條列出流程階段的常見問題
    \item 提供關鍵字搜尋功能
    \item 文件會條列出相關資料
    \item 文件會條列出各個領域中常見的問題
    \item 文件會條列出相關工具的資料
    \item 提供關鍵字搜尋功能 (搜尋google還是在文件內搜尋?)
  \end{enumerate} \\
  \hline
  測試方案:
  \begin{enumerate}
    \item 點開文件,確認是否有列出常見問題
    \item 點開文件,確認是否有列出相關資料
    \item 輸入不存在的關鍵字,確認是否有顯示沒有找到相關問題
    \item 點開某個Q\&A確認,內容是否與點擊的相同
  \end{enumerate} \\
  \hline
\end{tabularx}

\subsection*{}
\fontsize{12}{20}\selectfont
\begin{tabularx}{\textwidth}{|X|}
  \hline
  代號:SCAF-US-07 需求分析、設計階段、開發環境設定、命名規範、git commit format規範 以文件形式儲存分析結果 \\
  \hline
  故事:作為一個使用者,當我在撰寫需求文件時,我能夠將我的文件處存在此軟體中,並在我需要時叫出它做修改。 \\
  \hline
  註記:
  \begin{enumerate}
    \item 將使用hackmd撰寫的文件,儲存在此軟體中
    \item 更改文件後,可以讓更改者決定是否發文件通知所有人
    \item 規範文件會有一個獨立的頁面
  \end{enumerate} \\
  \hline
  測試方案:
  \begin{enumerate}
    \item 點開文件,確認是否內容相同
    \item 新增一個文件
    \item 編輯文件
    \item 刪除文件
    \item 更改文件後,選擇發送通知,確認每一個人都有收到通知
  \end{enumerate} \\
  \hline
\end{tabularx}

\subsection*{}
\fontsize{12}{20}\selectfont
\begin{tabularx}{\textwidth}{|X|}
  \hline
  代號:SCAF-US08 看板 \\
  \hline
  故事:作為一個使用者,我可以利用看板功能來紀錄與查看專案各個任務的開發進度,使團隊的協作與交付流程更加
  順暢。 \\
  \hline
  註記:
  \begin{enumerate}
    \item 任務可設定標籤、進度、成員、描述、時間排程
    \item 將任務放入看板中並依照任務的狀態分類,如:待處理、進行中、已完成
    \item 可對任務進行篩選查看
    \item 會發送訊息通知所有人,當有人將任務拖曳至看板中或任務狀態改變時
    \item 不同標籤的任務會以顏色區分
    \item 看板功能會有一個獨立的頁面
  \end{enumerate} \\
  \hline
  測試方案:
  \begin{enumerate}
    \item 新增一個任務至看板中,確認是否有顯示與發送通知
    \item 變更任務狀態,確認是否有顯示與發送通知
    \item 刪除一個任務,確認是否有顯示與發送通知
    \item 新增兩個不同標籤的任務,確認顏色是否相異
    \item 新增兩個相同標籤的任務,確認顏色是否相同
  \end{enumerate} \\
  \hline
\end{tabularx}

\subsection*{}
\fontsize{12}{20}\selectfont
\begin{tabularx}{\textwidth}{|X|}
  \hline
  代號:SCAF-US-09 google日曆API接口 \\
  \hline
  故事:作為一個使用者,我可以將看板任務的時間排程,同步到google日曆中,以便查看與管理時間。 \\
  \hline
  註記:
  \begin{enumerate}
    \item 可點擊授權按鈕,驗證完成後,將看板中所負責的任務同步到google日曆中
    \item 新增或變更任務時間排程後,會自動寄信至信箱中提醒負責人,並同步到負責人的google日曆中
    \item google日曆API接口功能會有一個獨立的頁面
    \item 輸入不合法的電子郵件,會出現錯誤訊息
  \end{enumerate} \\
  \hline
  測試方案:
  \begin{enumerate}
    \item 點擊授權按鈕,確認是否有成功授權
    \item 驗證過後,確認是否有將所負責的任務同步到google日曆中
    \item 新增任務,確認是否同步到google日曆中
    \item 變更任務時間排程,確認是否有同步到google日曆中
    \item 刪除任務,確認是否有同步到google日曆中
  \end{enumerate} \\
  \hline
\end{tabularx}

\subsection*{}
\fontsize{12}{20}\selectfont
\begin{tabularx}{\textwidth}{|X|}
  \hline
  代號:SCAF-US-10 Q\&A 文件顯示與導航 \\
  \hline
  故事:作為一個使用者,當我在Q\&A文件中,可以點選連結,導航到不同項目的Q\&A文件中,或是透過搜尋功能,找
  到相關的資料。 \\
  \hline
  註記:
  \begin{enumerate}
    \item 目錄中的連結可以導航到不同項目的Q\&A文件中
    \item 提供關鍵字搜尋功能
    % \item 提供模糊搜尋功能
    \item Q\&A文件是階層式的架構
  \end{enumerate} \\
  \hline
  測試方案:
  \begin{enumerate}
    \item 點開Q\&A文件,確認是否有列出目錄
    \item 點開目錄中的連結,確認是否有導航到不同項目的Q\&A文件中
    % \item 輸入空白的關鍵字,確認是否有顯示請輸入關鍵字
    \item 未輸入關鍵字就點擊搜尋,要顯示「請輸入關鍵字」
    \item 關鍵字搜尋輸入存在的關鍵字,確認是否有顯示相關資料
    \item 關鍵字搜尋輸入不存在的關鍵字,確認是否有顯示沒有找到相關資料
    % \item 模糊搜尋輸入關鍵字,確認是否有顯示相似的相關問題
  \end{enumerate} \\
  \hline
\end{tabularx}


\subsection*{3. 操作概念(Operation
 Concept)}

\begin{enumerate}[label=(\Alph*)]
  \item 會員註冊: 小林進入了這個 SCAF 平台,輸入名稱、信箱、密碼,註冊了一個帳號。 \\
  \includegraphics[width=\textwidth]{assets/wireframe/sign_up.png}
  \item 會員登入: 小林辦好帳號後,馬上試著登入平台來開始自己的軟體開發流程。 \\
  \includegraphics[width=\textwidth]{assets/wireframe/Login.png}
  \item 忘記密碼: 登入時卻發現,輸入好多次密碼都是錯的,只好重設密碼,輸入信箱之後,查看了自己的信箱,並點擊 Link 去重新設定密碼。\\
  \includegraphics[width=\textwidth]{assets/wireframe/forgot_password.png}
  \item 新建專案: 小林進入系統之後立刻想要開始第一個專案,於是按下了+號鍵。\\
  \includegraphics[width=\textwidth]{assets/wireframe/My_Projects.png}
  \item 專案初始化: 小林將自己的第一個專案命名為dian,並將專案設成Public,讓大家都能看見自己的專案,並加入README方便第一次進入專案的人,可以瞭解這個專案在做什麼和使用的流程。\\
  \includegraphics[width=\textwidth]{assets/wireframe/init_project.png}
  \item 建立repository: 小林因為想將前端和後端分開來放,所以按下new repository,建立兩個repository,repository3拿來放其他資料,分別在裡面放置前端和後端的檔案。\\
  \includegraphics[width=\textwidth]{assets/wireframe/My_Projects_repositories.png}
  \item 在repository新增file: 小林使用HTML建立好了頁面的基本框架,repository1中可以點擊Add file,將自己的檔案上傳到網頁上。\\
  \includegraphics[width=\textwidth]{assets/wireframe/My_projects_add_file.png }
  \item 需求、設計、開發環境設定、命名規範、git commit format文件: 小林現在想要來建立文件,於是他按下了documents,然後再document頁面中選擇需求文件,並按下edit(若是檔案已經存在,會顯示預覽畫面)。\\
  \includegraphics[width=\textwidth]{assets/wireframe/My_Projects_documents_requirement_dev_environment_naming_convention_git commit format.png }
  \item hackmd: 小林進入到hackmd中編輯需求文件。\\
  \includegraphics[width=\textwidth]{assets/wireframe/hackmd.png}
  \item 看板: 小林這時想記錄每個工作的完成進度,於是點擊kanban按鈕進入到kanban頁面,按下New task建立新的task,並設定任務標籤、任務描述、dayline與分配負責成員。\\
  \includegraphics[width=\textwidth]{assets/wireframe/My_Projects_kanban.png}
  \item 更改專案名稱: 小林這時發現自己的專案名稱太中二了,於是按下setting按鈕(project bar)進入到setting頁面,並點選general去更新自己的專案名稱。\\
  \includegraphics[width=\textwidth]{assets/wireframe/My_Projects_setting_general.png}
  \item 變更專案成員: 小林因為發現自己的組員都在耍廢,好像做完這份專案分錢的人太多了,於是他按下setting把所有人都踢出了,並按下invite邀請了一個大腿。\\
  \includegraphics[width=\textwidth]{assets/wireframe/My_Projects_setting_members.png}
  \item 處理加入專案申請: 小林覺得自己的專案很牛逼,於是想看看有沒有人提出申請加入專案,按下了pending invitation來看看有哪些有用的人,並按下YES確認了他們的申請。\\
  \includegraphics[width=\textwidth]{assets/wireframe/My_Projects_setting_pending_invitation.png}
  \item 刪除專案: 小林覺得這個專案真的是太麻煩了,於是按下了delete project把整個專案給刪除了。\\
  \includegraphics[width=\textwidth]{assets/wireframe/My_Projects_setting_pending_invitation.png}
  \item QA文件: 小林想要看看測試和部屬階段有沒有什麼常見的疑問,於是按下了QA,並按下testing or deployment。\\
  \includegraphics[width=\textwidth]{assets/wireframe/QA_testing_deployment.png}
  \item gitbook: 小林可以直接在下方的頁面中瀏覽gitbook。 \\
  \includegraphics[width=\textwidth]{assets/wireframe/gitbook.png}
  \item 上傳自己的傳片: 小林覺得自己太帥了想要讓大家都看到自己的照片,所以按下了main bar中的setting按鈕,並按下ADD your photo新增自己的照片。\\
  \includegraphics[width=\textwidth]{assets/wireframe/setting.png}
  \item 更改自己的名稱: 小林因為覺得中二的專案名稱被刪掉了很可惜,所以想要把自己的名稱搞的中二一點來平衡一下。\\
  \includegraphics[width=\textwidth]{assets/wireframe/setting.png}
  \item 授權帳號: 小林想利用自己的Google日曆來分配自己開發階段的期限,於是按下授權鍵給Google日曆授權。\\
  \includegraphics[width=\textwidth]{assets/wireframe/setting.png}
\end{enumerate}

\subsection*{4. 功能需求(Feature Requirements)}               

\begin{tabularx}{\textwidth}{
  |p{\dimexpr.2\linewidth-2\tabcolsep-1.33333\arrayrulewidth}%
  |p{\dimexpr.2\linewidth-2\tabcolsep-1.33333\arrayrulewidth}%
  |p{\dimexpr.6\linewidth-2\tabcolsep-1.33333\arrayrulewidth}|%
}
  \hline
  功能編號 & 功能名稱 & 功能說明 \\ \hline
  SCAF-FR-01 & 登入 & 使用者可以用 email 跟密碼進行登入,或是使用 Google 和 Github \\ \hline
  SCAF-FR-02 & 註冊 & 使用者可以用 email 註冊新帳戶。 \\ \hline
  SCAF-FR-03 & 忘記密碼 & 使用者可以發送驗證碼至 email 並重設密碼。 \\ \hline
  SCAF-FR-04 & 共用專案 & 使用者可以分享連結讓多人參與共同編輯。 \\ \hline
  SCAF-FR-05 & 開發流程模式 & 使用者可以選擇自己想要的設計流程。 \\ \hline
  SCAF-FR-06 & 專案初始化 & 使用者可以生成前後端基本的文件,添加外部的插件。 \\ \hline
  SCAF-FR-07 & 需求分析 & 使用者可以填入與客戶商討後的需求。 \\ \hline
  SCAF-FR-08 & 設計文件 & 使用者可以填入 API 形式和資料庫規範。 \\ \hline
  SCAF-FR-09 & 設定開發環境 & 使用者可以在 Docker 中實作。 \\ \hline
  SCAF-FR-10 & 命名規範 & 使用者可以在命名錯誤時修正。 \\ \hline
  SCAF-FR-11 & 風格規範 & 方便閱讀他人 code。 \\ \hline
  SCAF-FR-12 & 版本控管 & 使用者可以利用 git 各項功能。 \\ \hline
  SCAF-FR-13 & 工作計畫排程 & 使用者可以利用看板、行事曆、留言板,來觀察專案的排程。 \\ \hline
  SCAF-FR-14 & 整合其他工具 & 使用者可以加入自己想要的模組或是平台。 \\ \hline
  SCAF-FR-15 & 自動化測試 & 使用者可以寫相關測試程式碼,commit 時能夠自動測試。 \\ \hline
  SCAF-FR-16 & 自動化部署 & 使用者可以用 CI/CD 工具 \\ \hline
  SCAF-FR-17 & 產生規範文件 & 依據設計文件產生說明文件 \\ \hline
  SCAF-FR-18 & 問答 & 使用者可以問題提交上來,系統引導使用者排解問題 \\ \hline
\end{tabularx}

\subsection*{5. 非功能需求(Non-functional Requirements)}

\begin{tabularx}{\textwidth}{
  |p{\dimexpr.3\linewidth-2\tabcolsep-2\arrayrulewidth}%
  |p{\dimexpr.7\linewidth-2\tabcolsep-2\arrayrulewidth}|%
}
  \hline
  功能編號 &  功能說明 \\ \hline
  SCAF-NFP-01 & 系統人數同時負荷 1000 人 \\ \hline
  SCAF-NFP-02 & 回應時間在 1 秒內 \\ \hline
  SCAF-NFP-03 & 使用 Https 加密傳輸 \\ \hline
  SCAF-NFP-04 & CSRF 防護 \\ \hline
  SCAF-NFP-05 & 防止 DDoS 攻擊 \\ \hline
  SCAF-NFP-06 & Q\&A 搜尋結果在 1 秒內回應 \\ \hline
\end{tabularx}

\end{document}