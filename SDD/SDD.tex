\documentclass{report}
\usepackage[margin=2cm, bottom=1.5cm]{geometry}
\usepackage{type1cm}
\usepackage{amssymb}
\usepackage[fleqn]{amsmath}
\usepackage{tikz}
\usepackage{multicol}
\usepackage{makecell}
\usepackage{tabularx}
\usepackage[shortlabels]{enumitem}
\setlength{\columnsep}{1pt}
\setlist[enumerate]{nosep}

\makeatletter
\newenvironment{myalign*}{\ifvmode\else\hfil\null\linebreak\fi
  \hspace*{-\leftmargin}\minipage\textwidth
  \setlength{\abovedisplayskip}{0pt}%
  \setlength{\abovedisplayshortskip}{\abovedisplayskip}%
  \start@align\@ne\st@rredtrue\m@ne}%
{\endalign\endminipage\linebreak}

\usepackage{xeCJK}
\setCJKmainfont{Noto Sans TC}

\begin{document}
\title{%
  \fontsize{40}{60}\selectfont
  SCAF  \\ % 
  \vspace*{2cm}%
  \fontsize{24}{30}\selectfont
  設計文件
}

\author{
  \fontsize{18}{28}\selectfont
  \begin{tabularx}{0.9\textwidth}{
    |p{\dimexpr.25\linewidth-6\tabcolsep-2\arrayrulewidth}%
    |p{\dimexpr.75\linewidth-6\tabcolsep-2\arrayrulewidth}|%
  }
    \hline
    \centering 專案名稱 & SCAF 開發輔助工具 \\
    \hline
    \centering 撰寫日期 & 2022 / 11 / 21 \\
    \hline
    \centering 發展者 & 簡蔚驊 \! 鄧暐宣 \! 余威霆 \! 林佳何 \! 唐劭賢 \\
    \hline
  \end{tabularx}
}
\date{}
\usetikzlibrary{automata, positioning, arrows}
\maketitle
\tikzset{every state, accepting/.style={double distance=2pt}}

\fontsize{12}{18}\selectfont

\section*{1. 系統模型與架構(System Model/System Architecture)}

\begin{obeylines}
\parindent=0pt
描述系統之架構。可用C4 model的Container diagram + Component diagram、UML之Component diagram或單純的Block diagram (方塊圖)表達此系統包含的模組,以及與模組之間的關係。
架構圖應與SRS一致,但可特別強調介面(Interface)與實際佈署之環境。
放C4所有的圖片
鄧
\end{obeylines}

\section*{2. 介面需求與設計(Interface Requirement and Design)}

\begin{obeylines}
\parindent=0pt
根據架構圖,描述模組間的介面資訊,包含介面名稱、介面提供者、介面使用者、連結方式、輸入資料、輸出資料與介面描述等。
APIIIIIIIIIIIIIIIIIIIIIIIIIIIIIIIIIIIIIIIIIIIIIIIIIIIIII
簡
\end{obeylines}

\section*{3. 流程設計(Process Design)}

\begin{obeylines}
\parindent=0pt
可用UML之Activity diagram / State Transition diagram或一般的程式流程圖描述所開發的系統流程,目的是運用流程圖或狀態圖設計整個系統的完整運作。
此部份應是操作概念的細部設計。
使用者的使用流程 之前的流程圖+細部設計
唐
\end{obeylines}

\section*{4. 使用者畫面設計(User Interface Design)}

\begin{obeylines}
\parindent=0pt
可用各種畫面設計方式,如發展網頁prototype、繪製PowerPoint投影片等,去設計你的主要系統畫面。此部份應是SRS中使用者介面分析的進階設計版本。
線框圖? 包含所有UI設計 版型顏色等等
CLI工具UI設計
何
\end{obeylines}

\section*{5. 資料設計(Data Design)}

\begin{obeylines}
\parindent=0pt
設計此系統的資料庫schema、檔案結構、XML schema、JSON (JavaScript Object Notation)物件等。
NoSQL 類似json格式
簡
\end{obeylines}

\section*{6. 類別圖設計(Class Diagram)}

\begin{obeylines}
\parindent=0pt
發展UML之class diagram (可對應C4 model中的Code diagram),明確設計系統中包含哪些類別,以及這些類別之間的關係。
若遇到非一般物件類別或特殊物件類別,如HTML、JavaScript、JSP、Servlet等,可用stereotype表達,如<<JavaScript>>、<<Servlet>>。
可把重點放在定義程式結構:訂出所有類別名稱、拉出所有類別關係,再加入重要的attibute/operation即可。而特殊的類別職責設計可加上文字說明。
	StarUML
	前後端都要
\end{obeylines}

\section*{7. 實作方案(Implementation Languages and Platforms)}

\begin{obeylines}
\parindent=0pt
說明系統之平台(如網站或行動App)
說明預計採用的程式語言或技術(如後端Java、node.js、PHP等,前端JS+Boostrap、Android、iOS等)
說明預計採用的框架(Framework,如前端React、Angular、Vue,後端Spring、SpringBoot等)、函式庫(如前端jQuery、d3.js,後端jsoup、iText)、Servless服務(Firebase)等。
輕輕鬆酥
\end{obeylines}

\section*{8. 設計議題(Design Issue)}

\begin{obeylines}
\parindent=0pt
將設計此系統過程中遭遇的議題紀錄下來,包含(1)議題內容、(2)可能解決方案(alternative solutions)、(3)最後解決方案(final solution)與理由(rationale)。
唐劭賢輕輕鬆酥
大家有問題都丟這邊ㄛ
\end{obeylines}

\end{document}
